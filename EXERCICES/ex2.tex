\begin{fex}
 Soit $\Omega$ un domaine non vide de $\mathbb{C}$ et soit $f$ application
holomorphe sur $\Omega$. Montrer que les conditions suivantes sont
équivalentes~:
\renewcommand{\theenumi}{\alph{enumi}}
\begin{enumerate}
\item $f$ est constante sur $\Omega$.
\item $\Re(f)$ est constante sur $\Omega$.
\item $\Im(f)$ est constante sur $\Omega$.
\item $|f|$ est constante sur $\Omega$.
\end{enumerate}
\end{fex}
On séparera partie réelle et partie imaginaire de $f$ en posant $f=P+iQ$
Il est clair que (a) implique les autres conditions. Si on suppose (b), par
application des conditions de Cauchy et en utilisant que $P$ est constant:
\[
 \pd{P}{x} = \pd{Q}{y} = 0 , \pd{P}{y}=-\pd{Q}{x} = 0
\]
$\Omega$ étant connexe, on en déduit $Q$ constante. Il s'agit en fait d'une
équivalence $(b) \Leftrightarrow (c)$, le même raisonnement s'appliquant à $Q$.
On montre donc en sus que $|f|^2=P^2+Q^2$ est aussi une constante.
Finalement, si l'on suppose $P^2+Q^2$ constant, les dérivées partielles de
cette application par rapport à $x,y$ donnent;
\[
 P\pd{P}{x}+Q\pd{Q}{x}= 0,\,  P\pd{P}{y}+Q\pd{Q}{y}= 0
\]
En utilisant les conditions de Cauchy:
\[
 P\pd{P}{x}-Q\pd{P}{y}= 0,\,  P\pd{P}{y}+Q\pd{P}{x}= 0
\]
Il s'agit d'un système linéaire homogène de déterminant $P^2+Q^2=|f|^2$. Si
cette quantité est nulle, alors $f$ est nulle et (a) est vérifiée. Sinon:
\[
 \pd{P}{x}=\pd{P}{y}=0
\]
prouvant que $P$ est une constante et donc aussi $Q$ car $(b) \Leftrightarrow
(c)$, soit $f$ constante.

\begin{fex}
Soit $\Omega$ un ouvert de $\C$ ne contenant pas $0.$ Soit $f=P + i Q \colon \Omega \to \C$ une application $\R$ dérivable. Montrer, en posant $z=re^{i\theta}$ pour $z \in \Omega$, que les conditions de Cauchy en un point $z_0 = r_0 e^{i \theta_0}$ sont équivalentes à:
\[
\begin{cases}
    \frac{\partial P}{\partial r}\vert_{r_0,\theta_0} = \frac{1}{r}\frac{\partial Q}{\partial \theta}\vert_{r_0,\theta_0} \\
    \frac{\partial P}{\partial \theta}\vert_{r_0,\theta_0} =- r\frac{\partial Q}{\partial r}\vert_{r_0,\theta_0}
\end{cases}
\]
\end{fex}
Il s'agit d'une application de la dérivation en chaîne. En posant $z = x + i y$, $f = P + i Q$ 
et en supposant vérifiées les conditions de Cauchy en coordonnées cartésiennes:
\begin{equation}
\begin{split}
    & \frac{\partial P}{\partial r} = \frac{\partial P}{\partial x}\frac{\partial x}{\partial r} +\frac{\partial P}{\partial y}\frac{\partial y}{\partial r}
    = \frac{\partial Q}{\partial y} \cos \theta -\frac{\partial Q}{\partial x} \sin \theta\\
    & \frac{\partial Q}{\partial \theta} = \frac{\partial Q}{\partial x}\frac{\partial x}{\partial \theta} + \frac{\partial Q}{\partial y}\frac{\partial y}{\partial \theta} = -\frac{\partial Q}{\partial x} r \sin\theta + \frac{\partial Q}{\partial y} r \cos \theta
\end{split}
\end{equation}
Par identification:
\[
\frac{\partial P}{\partial r} = \frac{1}{r}\frac{\partial Q}{\partial \theta}
\]
De même:
\begin{equation}
\begin{split}
    & \frac{\partial P}{\partial \theta} = \frac{\partial P}{\partial x}\frac{\partial x}{\partial \theta} +\frac{\partial P}{\partial y}\frac{\partial y}{\partial \theta}
    = - \frac{\partial Q}{\partial y} r \sin \theta -\frac{\partial Q}{\partial x} r \cos \theta\\
    & \frac{\partial Q}{\partial r} = \frac{\partial Q}{\partial x}\frac{\partial x}{\partial r} + \frac{\partial Q}{\partial y}\frac{\partial y}{\partial r} = \frac{\partial Q}{\partial x}  \cos \theta + \frac{\partial Q}{\partial y} \sin \theta
\end{split}
\end{equation}
d'où:
\[
\frac{\partial P}{\partial \theta} = -r \frac{\partial Q}{\partial r}
\]