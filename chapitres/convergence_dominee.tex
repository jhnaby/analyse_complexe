\chapter{Convergence dominée}
La puissance de la théorie de l'intégration de Lebesgue réside dans
les théorèmes de convergence que l'on peut en déduire. Historiquement,
la première théorie de l'intégration formalisée est celle de Riemann qui a été
présentée en introduction. Son défaut principal est la nécessité d'avoir
 la convergence uniforme pour intervertir limite et intégrale. Dans le cadre de
 la théorie de Lebesgue, la convergence simple suffit, en ajoutant, dans le cas
 d'applications de signe quelconque, une hypothèse aisée à vérifier. Une conséquence
importante des théorèmes qui vont être maintenant établis sera l'existence
d'espaces d'applications sommables qui pourront être munis d'une structure
d'espace de Banach.

\begin{mandatory}
\begin{theorem}\label{ch2:mono} Convergence monotone positive. Soit $(f_n)_{n \in \mathbb{N}}$ une suite croissante d'applications
mesurables positives sur $(E, \mathcal{T}, \mu)$. 
Sa limite simple $f$ existe, est mesurable et
l'on a~:
\[
\int_E f d \mu = \lim_{n \to +\infty} \int_E f_n d \mu
\]
\end{theorem}
\end{mandatory}
\begin{proof}
L'existence de $f$ ne pose aucun problème, la suite $(f_n)$ étant
supposée croissante (il est possible que $f$ prenne la valeur
$+\infty$ sur une partie de $E$). La mesurabilité de $f$ découle de la
mesurabilité d'une limite simple d'applications mesurables. 
Pour tout entier $n$, soit $(\phi_{n,k})_{k \in \mathbb{N}}$ une
suite croissante d'applications étagées convergeant simplement vers
$f_n$ (une telle suite existe en vertu de la proposition
\ref{ch2:1}). On pose pour tout entier $n \in \mathbb{N}, g_n =
sup_{k=0 \dots n} \phi_{k,n}$. $g_n$ est étagée et la suite $(g_n)$
est croissante. Par ailleurs, pour tout entier $k$, on vérifie
aisement que~:
\[
f_k \leq \lim_n g_n \leq f
\]
d'où, en faisant tendre $k$ vers $+\infty$, $\lim_n g_n = f$. Par la
proposition \ref{ch2:2}, on obtient~:
\[
\int_E f d \mu =  \lim_n \int_E g_n d \mu
\]
Par définition de $g_n$, $\int_E g_n d \mu \leq \int_E f_n d \mu$, d'où~:
\[
\int_E f d \mu = \lim_n \int_E g_n d \mu \leq \lim_n \int_E f_n d \mu
\]
comme l'inégalité $\int_E f d \mu \geq \int_E f_n d \mu$ est évidente,
on en déduit l'égalité.
\end{proof}
\begin{rem}
L'argument diagonal utilisé dans cette démonstration est un grand classique des
mathématiques: il faut le connaître et savoir l'utiliser!
\end{rem}
\begin{mandatory}
\begin{theorem}\label{ch2:beppo} Beppo-Levi.
Soit $(f_n)_{n \in \mathbb{N}}$ une suite croissante d'applications
sommables telle qu'il existe $M >0$ avec~:
\[
\forall n \in \mathbb{N}, \, \int_E |f_n|d \mu \leq M
\] 
alors $(f_n)$ converge presque partout vers une application sommable
$f$ et~:
\[
\lim_n \int_E f_n d \mu = \int_E f d\mu
\]
\end{theorem}
\end{mandatory}
\begin{proof}
On remarque tout d'abord que la suite $(f_n)$ étant croissante, elle
admet une limite simple $f$ (pouvant éventuellement être infinie sur
une partie de $E$).
Pour tout $n$, l'application $g_n = f_n -f_0$ est positive sommable car~:
\[
\int_E |f_n - f_0| d \mu \leq 2 M
\]
Le théorème \ref{ch2:mono} s'applique à la suite $(g_n)$ d'où~:
\[
\lim_n \int_E g_n d \mu = \int_E f - f_0 d \mu
\]
d'où~:
\[
\lim_n \int_E f_n d \mu = \int_E f d \mu
\]
Mais par ailleurs $\forall n \in \mathbb{N}, \int_E |f_n| d \mu \leq
M$ et donc $\int_E |f| d \mu$ est finie. On en déduit alors que la
partie de $E$ sur laquelle $f$ n'est pas finie est de mesure nulle, ce
qui termine la preuve.
\end{proof}
\begin{mandatory}
\begin{theorem}\label{ch2:fatou} Lemme de Fatou.
Soit $(f_n)_{n \in \mathbb{N}}$ une suite d'applications mesurables non
  négatives. On a~:
\[
\int_E (\liminf_n f_n) d \mu \leq \liminf_n \left(\int_E f_n d \mu \right)
\]
\end{theorem}
\end{mandatory}
\begin{proof}
Pour tout entier $n$, soit $g_n = \inf_{k \geq n} f_k$. On a~:
\[
\liminf_n f_n = \lim_n g_n
\]
La suite $(g_n)$ vérifie les hypothèse du théorème \ref{ch2:mono}, on
a donc~:
\[
\int_E \liminf_n f_n d\mu = \lim_n \int_E g_n d\mu \leq \liminf_n
\int_E f_n d \mu
\]
\end{proof}
\begin{mandatory}
\begin{theorem}\label{ch2:domi} Convergence dominée.
Soit $(f_n)$ une suite d'applications mesurables convergeant
 simplement $\mu$-presque partout vers une application
$f$ et telles qu'il existe
une application sommable $g$ avec~:
\[
\forall n \in \mathbb{N}, \, |f_n| \leq g \quad \mu\text{-pp}
\]
Alors~:
\begin{itemize}
\item $f$ est sommable.
\item $\lim_n \int_E f_n d\mu  = \int_E f d\mu$.
\item $\lim_n \int_E |f_n -f | d \mu = 0$.
\end{itemize}
\end{theorem}
\end{mandatory}
\begin{proof}
On peut supposer que la convergence simple a lieu partout, le
comportement sur un ensemble de mesure nulle ne modifiant en rien les
intégrales. 
Les applications $f_n$ sont sommables car~:
 \[\int_E |f_n| d \mu \leq
\int_E g d \mu < +\infty \]
Par passage à la limite, $|f| \leq g$ d'où la sommabilité de $f$.
En remarquant que pour tout $n$, $|f_n - f| \leq 2g$, le lemme de
Fatou permet d'écrire~:
\[
\liminf_n \int_E (2g -|f_n-f|) d \mu \geq \int_E \liminf_n (2g - |f_n-f|)
d \mu
\]
par hypothèse, $f_n \to f$ d'où~:
\[
 \liminf_n \left (2g - |f_n-f| \right ) = 2g
\]
La linéarité de l'intégrale donne alors~:
\[
\liminf_n \int_E (2g -|f_n-f|) d \mu = 2 \int_E g d \mu - \limsup_n
\int_E |f_n-f| d \mu = 2 \int_E g d \mu
\]
soit~:
\[
 \limsup_n \int_E |f_n-f| d \mu = 0
\]
ce qui prouve le point 3) du théorème.
Comme~:
\[
\left | \int_E f_n d \mu - \int_E f d \mu \right | \leq \int_E |f_n
-f| d \mu
\]
le point 2) en découle immédiatement.
\end{proof}
\begin{rem}
\begin{itemize}
\item Les conclusions du théorème de convergence dominée (ainsi que des
précédents) s'appliquent si l'on remplace les suites d'applications
par des familles $(f_\lambda)_{\lambda \in I}$ avec $I$ un ensemble
d'indice muni d'une structure d'espace métrique. 
\item Il est très important de prendre garde au fait que l'application
  $g$ ne peut en aucun cas dépendre de $n$~: c'est une source
  fréquente d'erreurs.
\end{itemize}
\end{rem}
Il convient de toujours bien vérifier l'hypothèse de domination, qui peut se
révéler délicate, comme l'exercice suivant va le montrer.
\begin{exercice}
\begin{itemize}
\item Soit la suite d'applications $(f_n)$ définies sur $[0,1]$ par~:
\[
\forall x \in [0,1], \, f_n(x) = \frac{nx}{1+n^2x^2}
\]
Déterminer la limite simple de cette suite. Conclure quant à la limite de~:
\[
\int_{[0,1]} f_n(x) d \lambda(x)
\]
\item On considère maintenant la suite $(g_n)$, applications définies sur $\mathbb{R}$ par~:
\[
g_n(x) = \frac{1}{n} 1_{[0, n]}
\]
Déterminer la limite simple de cette suite. Peut-on appliquer le théorème de convergence
dominée à~:
\[
\int_{\mathbb{R}} g_n(x) d\lambda(x)
\]
\end{itemize}
\end{exercice}
\section{Applications}
\subsection{Intégrales dépendant d'un paramètre}
\begin{prop}\label{ch2:contpar} Continuité des intégrales dépendant
  d'un paramètre.
Soit $(M,d)$ un espace métrique, $\lambda_0 \in M$, $(E, \mathcal{T},\mu)$ un espace mesuré et soit $F : M \times E \to
\mathbb{R}$ une application telle que~:
\begin{itemize}
\item Pour tout $\lambda \in M$, l'application $x \to f(\lambda,x)$
  est mesurable.
\item L'application $\lambda \to f(\lambda,x)$ est continue en
  $\lambda_0$ pour $\mu$-presque tout $x$.
\item Il existe une application sommable $g$ et un voisinage $V$ de
  $\lambda_0$ tels que~:
\[
\forall \lambda \in V, \, |F(\lambda,x)| \leq g(x)
\]
pour $\mu$-presque tout $x$.
alors l'application~:
\[
\lambda \in V \to \int_E F(\lambda,x) d\mu(x)
\]
est bien définie et est continue en $\lambda_0$.
\end{itemize}
\end{prop}
\begin{proof}
La sommabilité de l'application $x \to F(\lambda,x)$ est acquise pour
$\lambda \in V$ par l'hypothèse de domination. Par ailleurs, soit
$(\lambda_n)$ une suite de points de $V$ de limite $\lambda_0$. Le
théorème de convergence dominée donne~:
\[
\lim_n \int_E F(\lambda_n,x)d\mu(x) = \int_E \lim_n  F(\lambda_n,x)d\mu(x) =
\int_E F(\lambda_0,x)d\mu(x)
\]
ce qui prouve la continuité en $\lambda_0$ de l'intégrale dépendant du
paramètre $\lambda$.
\end{proof}
\begin{prop}\label{ch2:derpar} Dérivabilité des intégrales dépendant
  d'un paramètre.
Soit $I$ un intervalle ouvert de $\mathbb{R}$, $y_0 \in I$, et soit $F : I \times E \to
\mathbb{R}$ une application telle que~:
\begin{itemize}
\item Pour tout $y \in I$, l'application $x \to F(y,x)$ est sommable.
\item L'application $y \to F(y,x)$ est dérivable en $y_0$ pour $\mu$-presque
  tout $x$. On notera~:
\[
\frac{\partial F}{\partial y}
\]
l'application égale à la dérivée de $F$ par rapport à $y$ en tout
point $x$ où elle est définie et prenant une valeur arbitraire
ailleurs.
\item Il existe une application sommable $g$ telle que pour tout $y
  \in I$~:
\[
|F(y,x)-F(y_0,x)| \leq g(x) |y-y_0|
\]
pour $\mu$-presque toute valeur de $x$.
\end{itemize}
Alors l'application~:
\[
\lambda \to \int_E F(y,x)d\mu(x)
\]
est dérivable en $y_0$ et sa dérivée vaut~:
\[
 \int_E \frac{\partial F(y_0,x)}{\partial y} d \mu(x)
\]
\end{prop}
\begin{proof}
La démonstration est une application immédiate du théorème de
convergence dominée à la suite~:
\[
\frac{F(y_n,x)-F(y_0,x)}{y_n -y_0}
\]
avec $y_n \to y_0$.
\end{proof}
\begin{rem}
Le théorème des accroissement finis montre que si il existe $g$
sommable telle que~:
\[
\forall y \in I, \, \left |
\frac{\partial F(y,x)}{\partial y} 
\right | \leq g(x)
\]
pour $\mu$-presque tout $x$, alors l'hypothèse 3) de la proposition
est automatiquement vérifiée. 
\end{rem}
\subsection{Sommabilité terme à terme}
Soit $(f_n)_{n \in \mathbb{N}}$ une suite d'applications mesurables
telle que~:
\[
\sum_{n \in \mathbb{N}} \int_E |f_n| d \mu < +\infty
\]
alors~:
\begin{itemize}
\item $\sum_{n \in \mathbb{N}} f_n$ converge absolument $\mu$-presque
  partout et est som\-mable.
\item On a~:
\[
\sum_{n \in \mathbb{N}} \int_E f_n d \mu = \int_E \sum_{n \in
  \mathbb{N}} f_n d \mu 
\]
\end{itemize}
\begin{proof}
Soit $g_n = \sum_{k = 0 \dots n} |f_n|$. La suite $g_n$ est
positive croissante, le théorème de convergence monotone s'applique
et~:
\[
\lim_n \int_E g_n d \mu = \int_E \lim_n g_n d \mu 
\]
Or par hypothèse~:
\[
\lim_n \int_E g_n d \mu = \sum_n \int_E |f_n| d \mu < +\infty
\]
la limite $g$ de la suite $g_n$ est sommable, donc $g$ est finie
$\mu$-presque partout. On en déduit la convergence absolue
$\mu$-presque partout de la serie de terme général $f_n$.
Par ailleurs, le théorème de convergence dominée s'applique aux sommes
partielles de la série de terme général $f_n$ en prenant comme application
dominante~:
\[
\sum_n |f_n|
\]
Ceci montre la seconde partie de la proposition.
\end{proof}
\begin{exercice}
Soit $f : \mathbb{R} \to \mathbb{R}$ une application sommable. On 
définit l'application $\phi : \mathbb{R} \to \mathbb{C}$ par~:
\[
\forall \xi \in \mathbb{R}, \, \phi(\xi) = \int_{\mathbb{R}} f(t) e^{-i 2 \pi \xi t} d \lambda(t)
\]
\begin{itemize}
\item Montrer que $\phi$ est bien définie et est continue sur $\mathbb{R}$.
\item On suppose maintenant que $f$ est telle que~:
\[
\int_{\mathbb{R}} | t f(t) | d \lambda(t) < +\infty
\]
Montrer que $\phi$ est dérivable et calculer sa dérivée.
\end{itemize}
\end{exercice}
On peut dans certains cas utiliser le théorème de convergence dominée pour
calculer la valeur d'intégrales généralisées (qui ne sont pas des intégrales au
sens strict, mais uniquement des limites). Un cas classique est la détermination
de l'intégrale de $\sin x / x$ sur $\mathbb{R}$, qui est un exercice
fréquemment traité en classes préparatoires.
\begin{exercice}
Soit l'application $f : \mathbb{R} \to \mathbb{R}$~:
\[
x \to \left \{
\begin{array}{cc}
\frac{\sin x}{x} & x > 0 \\
1 & x = 0 \\
0 & x < 0
\end{array}
\right .
\]
\begin{itemize}
\item Montrer que $f \notin L^1, f \in L^2$.
\item Montrer que pour tout réel $a > 0$ l'application~:
\[
F : a \to \int_{\mathbb{R}} f(x) e^{-ax} dx
\]
est bien définie et est dérivable sur $\mathbb{R}^{+*}$.
\item Calculer $lim_{a \to +\infty} F(a)$ et en déduire une expression simple de $F$.
\item Montrer que $F$ se prolonge par continuité à droite en 0 et calculer $F(O^+)$.
\end{itemize}
La valeur obtenue pour $F(0^+)$ est aussi celle de l'intégrale généralisée de
$\sin x/ x$ sur $\mathbb{R}^+$.
\end{exercice}
\subsection{Intégrale de Riemann}
L'intégrale au sens de Riemann d'une application bornée sur un compact est,
lorsqu'elle existe, de m\^eme valeur que celle de Lebesgue qui  existe
nécessairement elle aussi. La suite de cette section va établir ce résultat.
 Dans cette partie, $f : [a,b] \to \mathbb{R}$ est une
application \textit{bornée}.
\begin{defn}
Une subdivision $\mathcal{P}$ de l'intervalle $[a,]$ est une suite de
réels $(x_i)_{i=0}^{N}$ telle que $a=x_0 < x_1 < \dots < x_{N-1} < x_N
= b$. Le pas d'une subdivision est~:
\[
\inf_{n=0\dots N-1} (x_{n+1} -x_n)
\]
Enfin, on dira qu'une subdivision $\mathcal{P}_1$ est plus fine qu'une
subdivision $\mathcal{P}_2$ si tout point de $\mathcal{P}_2$
appartient aussi à $\mathcal{P}_1$.
\end{defn}
\begin{defn}
Une application en escalier sur $[a,b]$ est une application de la
forme~:
\[
\sum_{n=0}^{N-1} y_n 1_{[x_n, x_{n+1}[}
\]
avec $(y_n)_{n=0\dots N-1}$ réels et $(x_n)$ subdivision de $[a,b]$.
\end{defn}
\begin{defn}
Soit $f= \sum_{n=0}^{N-1} y_n 1_{[x_n, x_{n+1}[}$ une application en
    escalier. L'intégrale de Riemann de $f$ est la quantité~:
\[
\int_a^b f(x) dx = \sum_{n=0}^{N-1} y_n (x_{n+1} -x_n)
\]
\end{defn}
\begin{defn}
Soit $f$ bornée sur $[a,b]$. L'intégrale inférieure (resp. supérieure)
de $f$ est~:
\[
\sup \left \{ \int_a^b h(x) dx, h \mbox{ en escalier }, h \leq f \right
\}
\]
resp.
\[
\inf \left \{ \int_a^b h(x) dx, h \mbox{ en escalier }, h \geq f \right
\}
\]
Si les deux quantités sont égales, on dit que $f$ est
Riemann-intégrable et l'intégrale de $f$, notée $\int_a^b f(x) dx$ est
la valeur commune.
\end{defn}
\begin{theorem}
Une application $f$ bornée sur $[a,b]$ est Riemann-intégrable si et
seulement si elle est égale presque partout (relativement à la mesure
de Lebesgue) à une application continue. On a en outre~:
\[
\int_a^b f(x) dx = \int_{[a,b]} f d \lambda
\]
avec $\lambda$ mesure de Lebesgue.
\end{theorem}
\begin{proof}
Il est facile de remarquer que toute application en escalier étant une
application étagée, les deux intégrales coïncident pour cette classe
d'applications. 
Supposons tout d'abord $f$ Riemann-intégrable.
Soit $(h_n)_{n \in \mathbb{N}}$ une suite d'applications en escaliers
 telle que $\forall n, h_n \geq f$ et
$\lim_n \int_a^b h_n(x) dx = \int_a^b f(x) dx$. On peut, sans perte de
généralité, supposer la suite $h_n$ décroissante.  Elle converge
vers une limite $h$ et par application du thèorème de convergence
dominée (on peut majorer $|h_n|$ par une constante qui est sommable
sur le compact $[a,b]$)~: 
\[
\int_{[a,b]} h d \lambda = \lim_n \int_{[a,b]} h_n d \lambda = \lim_n \int_a^b
h_n(x) dx = \int_a^b f(x) dx
\]
De même, on peut choisir une suite croissante d'applications en
escalier $(g_n)$, telle que $g_n \leq f$ pour tout $n$ et $\lim_n
\int_a^b g_n(x) dx = \int_a^b f(x) dx$. Le théorème de convergence
dominée s'applique à nouveau et l'on a~:
\[
\int_{[a,b]} g d \lambda = \int_a^b f(x) dx
\]
avec $g$ limite de la suite $(g_n)$. 
On en déduit~:
\[
\int_{[a,b]} (h-g) d \lambda = 0
\]
Soit $h(x)=g(x)=f(x)$ pour $\lambda$-presque toute valeur de
$x$. Comme $h$ et $g$ sont des limites de fonctions en escalier, elles
ne peuvent présenter qu'un ensemble dénombrable de points de
discontinuité, d'où $f$ est égale  $\lambda$-presque partout à une
application continue et l'intégrale de Riemann coïncide avec celle de Lebesgue.
Supposons maintenant que l'on se donne $f$ bornée, égale
$\lambda$-presque partout à une application continue. Soit
$\mathcal{P}_n$ la partition de $[a,b]$ qui comporte $2^n + 1$ points
$x_{i,n}, i=0 \dots 2^n$ divisant $[a,b]$ en segments égaux. Soit
$M_{i,n}$ le maximum (resp. $m_{i,n}$ le minimum) de$f$ sur
chaque intervalle $[x_{i,n}, x_{i+1,n}]$. On pose $g_n =
\sum_{i=0}^{2^n-1} m_{i,n} 1_{[x_{i,n}, x_{i+1,n}[}$,
    $h_n=\sum_{i=0}^{2^n-1} M_{i,n} 1_{[x_{i,n}, x_{i+1,n}[}$.
	$f$ étant égale presque partout à une application continue, les suites
	$(g_n),(h_n)$ convergent $\lambda$-presque partout vers $f$ et
	par application du théorème de convergence dominée~:
\begin{align*}
\int_{[a,b]} f d \lambda = & \lim_n \int_{[a,b]} g_n d\lambda = \lim_n
\int_a^b g_n(x) dx \\  
= & \lim_n \int_{[a,b]} h_n d\lambda = \lim_n \int_a^b h_n(x) dx
\end{align*}
L'application $f$ est donc Riemann-intégrable et les deux intégrales
sont égales.
\end{proof} 
Ce résultat montre que, sur un compact, on peut se permettre d'utiliser
l'une ou l'autre intégrale dès lors que l'application est égale
presque partout à une application continue. Ceci induit à confondre
par abus de notation $\int_a^b$ avec $\int_{[a,b]}$.
Toute application $f$ qui est limite uniforme d'applications en escalier admet
une primitive $F$. On peut utiliser le résultat précédent pour calculer dans ce
cas l'intégrale de Lebesgue sur un compact $[a,b]$ à l'aide de $F$:
\[
\int_{[a,b]}f(t) d \lambda(t) = F(b) - F(a)
\]

\begin{rem}
Il convient d'être d'une extrême prudence avec ce que l'on appelle les
intégrales généralisées !
Dans le cas d'intervalles infinis (semi-infinis) ou d'ouverts, la
notation~:
\[
\int_a^b f(x) dx
\]
désigne, si elle existe, la limite~:
\[
\lim_{x\to a^+, y \to b^-} \int_x^y f(x) dx
\]
Il ne s'agit donc pas, à proprement parler, d'une intégrale de
Riemann. Dans le cadre de la théorie de Lebesgue, il n'y a pas de
problème de définition même si l'on n'intègre pas sur un compact, et la
notion d'intégrale généralisée n'a pas de sens. Dans ce cas, il
convient donc de bien distinguer les notations. Cependant, si une
application $f$ est telle que $\int_a^b |f(x)|dx$ existe (on dit alors
que $f$ admet une intégrale généralisée absolument convergente),
l'intégrale de Lebesgue de $f$ existe sur le même domaine et avec la
même valeur (application immédiate du théorème de convergence dominée).
\end{rem}