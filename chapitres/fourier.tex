\chapter{Transformation de Fourier}
\section{Introduction}
La transformation de Fourier a été introduite en 1822 par J. Fourier au
cours de ses travaux sur l'équation de la chaleur. Son assertion initiale était
que toute application périodique pouvait se décomposer sous la forme d'une série
de fonctions sinus (ou cosinus), de fréquences multiples de l'inverse de la
période de l'application. Bien qu'inexacte en général, cette écriture a ouvert
la voie à une famille de représentation de fonctions éventuellement discontinues
sous la forme d'une série d'applications elles-même très régulières. En
interprétant les sommes de Fourier comme des sommes de Riemann, on peut
également étendre la théorie à une transformation intégrale, d'une grande
importance aussi bien pour les applications (traitement du signal en
particulier) que du point de vue fondamental (équations aux dérivées
partielles). La grande force de la transformation de Fourier est de permettre
une reformulation des opérateurs différentiels en opérateurs effectuant des
produits par des applications polynomiales: résoudre une équation différentielle
dans ce cadre revient à résoudre une équation ordinaire (que l'on espère plus
simple). 
\section{Transformée de Fourier des applications de $L^1$}
\begin{mandatory}
\begin{defn}
Soit $f : \mathbb{R}^n \to \mathbb{C}$ une application de
$L^1(\mathbb{R}^n)$. La transformée de Fourier de $f$ est
l'application $\widehat{f} : \mathbb{R}^n \to \mathbb{C}$ définie
par~:
\[
\forall \xi \in \mathbb{R}^n, \widehat{f}(\xi) = \int_{\mathbb{R}^n}
f(x) e^{-i \langle x ,\xi \rangle} d \lambda(x)
\]
\end{defn}
\end{mandatory}
\begin{rem}
Dans certains cas, on trouve une définition de la transformation de
Fourier faisant apparaître un facteur $2\pi$ devant le produit
scalaire~: ceci ne change rien aux propriétés fondamentales, sauf en
ce qui concerne la formule d'inversion que nous verrons plus loin. Cette
formulation alternative est la plus courant en traitement du signal, où l'on
recherche une interprétation de la variable de Fourier en terme de fréquence.
\end{rem}
On utilisera dans la suite indifféremment la notation $\widehat{f}$ ou
$\mathcal{F}(f)$ pour désigner la transformée de Fourier de $f$.
\begin{mandatory}
\begin{prop}
La transformée de fourier d'une application de $L^1(\mathbb{R}^n)$ est
une application de $L^\infty(\mathbb{R}^n)$. La transformation de
Fourier est un opérateur linéaire continu entre ces deux ensembles.
\end{prop}
\end{mandatory}
\begin{proof}
On a de façon évidente~:
\[
\forall \xi \in \mathbb{R}^n, \, |\widehat{f}(\xi)| \leq
\int_{\mathbb{R}^n} | f(x) | d\lambda(x)
\]
d'où~:
\[
\| \widehat{f} \|_{\infty} \leq \| f \|_1
\]
ce qui montre la continuité de l'opérateur linéaire défini par la
transformée de Fourier.
\end{proof}
\begin{exercice}
Déterminer les transformées de Fourier des applications suivantes~:
\[
\begin{array}{c}
f_1 \colon x \mapsto e^{-|x|} \\
f_2 \colon x \mapsto \begin{cases}
1-|x| & \text{ si } |x| \leq 1 \\
0 & \text{ sinon }
\end{cases}
\end{array}
\]
\end{exercice}
Le lemme suivant est important pour établir les formules d'inversion de la
transformation de Fourier:
\begin{mandatory}
\begin{prop}\label{fourier:1}
Soient $f,g \in L^1(\mathbb{R}^n)$. On a~:
\[
\int_{\mathbb{R}^n} f(x) \widehat{g}(x) d \lambda(x) = \int_{\mathbb{R}^n} \widehat{f}(y) g(y) d \lambda(y)
\]
\end{prop}
\end{mandatory}
\begin{proof}
Les applications $f, g$ étant dans $L^1(\mathbb{R}^n)$,
l'application $(x,y) \to f(x)g(y)$ est dans $L^1(\mathbb{R}^{2n})$. 
Pour prouver ce fait, on note tout d'abord que l'application $(x,y) \mapsto
f(x)$ est mesurable car composée de la projection sur le premier facteur et de
l'application $f$. De même pour l'application $(x,y) \mapsto g(y)$. On en déduit
que l'application: $(x,y) \mapsto f(x)g(y)$ est mesurable car produit
d'applications mesurables. Le théorème de Tonnelli permet alors d'écrire:
\[
\int_{\mathbb{R}^{2n}}|f(x)g(y)|d\lambda(x)d\lambda(y) = \int_{\mathbb{R}^n}
|f(x)| d \lambda(x)\int_{\mathbb{R}^n}
|g(y)| d \lambda(y) < +\infty
\]
 On en
déduit par application du théorème de Fubini~:
\begin{align*}
& \int_{\mathbb{R}^n} f(x) \int_{\mathbb{R}^n} g(y) e^{-i \langle x,y
      \rangle} d \lambda(y) d  \lambda(x) = \\
& \int_{\mathbb{R}^{2n}} f(x) g(y) e^{-i \langle x,y
      \rangle} d \lambda (x,y) = \\
&\int_{\mathbb{R}^n} g(y) \int_{\mathbb{R}^n} f(x) e^{-i \langle x,y
      \rangle} d \lambda(x) d  \lambda(y) 
\end{align*}
ce qui montre la proposition.
\end{proof}
La transformation de Fourier posséde un comportement simple vis à vis des
opérations de translation et de changement d'échelle. Le résultat ci-dessous est
donné sans démonstration, que le lecteur pourra faire à titre d'exercice.
\begin{mandatory}
\begin{prop}
Soit $f \in L^1(\mathbb{R}^n)$. Soit $v \in \mathbb{R}^n$. On définit la
translatée de $f$ en $v$ par:
\[
T_v f \colon x \in \mathbb{R}^n \mapsto f(x-v)
\]
On a alors:
\[
\forall \xi \in \mathbb{R}^n, \, \widehat{T_v f}(\xi) = e^{-i \langle
v, \xi \rangle}\widehat{f}(\xi)
\]
De même, soit $t > 0$ un réel. La dilatée de $f$ par $t$ est l'application:
\[
S_t f \colon x \in \mathbb{R}^n \mapsto t^{-n} f\left(\frac{x}{t}\right)
\]
On a:
\[
\forall \xi \in \mathbb{R}^n, \, \widehat{S_t f}(\xi) = \widehat{f}(t \xi)
\]
\end{prop}
\end{mandatory}
Enfin, comme annoncé en introduction, une dérivée partielle sera transformée en
multiplication par un monôme.
\begin{term}
On notera $D_jf$  la dérivée partielle de $f$ par rapport à la
$j$-ième coordonnée.
\end{term}
\begin{mandatory}
\begin{prop}
Soit $j \in \{ 1 \dots n \}$.
Soit $f \in L^1(\mathbb{R}^n)$ telle que l'application $x \to x_j
f(x)$ (avec $x_j$ $j$-ième composante du vecteur $x$) soit dans
$L^1(\mathbb{R}^n)$ alors~:
\[
D_j \widehat{f} = - i \widehat{x_j f(x)}
\]
\end{prop}
\end{mandatory}
\begin{proof}
La transformée de Fourier de $f$ est définie par~:
\[
\widehat{f}(\xi) = \int_{\mathbb{R}^n} f(x) e^{- i \langle x, \xi
  \rangle } d \lambda(x)
\]
Il s'agit d'une intégrale dépendant d'un paramètre. Comme par ailleurs
on a pour tout $x_0 \in \mathbb{R}^n$ et tout réel $h$~:
\[
|f(x) e^{-i \langle x, \xi  \rangle } - f(x) e^{-i \langle x, \xi + h
 e_j  \rangle }  |\leq |f(x)||x_j h|
\]
Le théorème de dérivation s'applique donc et donne le résultat.
\end{proof}
\begin{lemme}
L'ensemble des applications continues à support compact est dense dans
$L^1(\mathbb{R}^n)$. 
\end{lemme}
\begin{proof}
Les applications étagées sont denses de façon évidente dans
$L^1(\mathbb{R}^n)$. Montrons dans un premier temps que l'espace
engendré par les applications de la forme~:
\[
1_{\prod_{i=1\dots n} ]a_i, b_i[}
\]
est dense dans $L^1(\mathbb{R}^n)$. Soit $A$ un borélien borné.
Par définition de
la mesure de Lebesgue, il existe, pour tout $\epsilon >0$, une suite 
d'hyperrectangles $R_j = \prod_{i=1\dots n} ]a_i(j), b_i(j)[$ telle que $\cup_j R_j \supset A$ et~:
\[
\sum_j \lambda(R_j) < \lambda(A) + \frac{\epsilon}{2}
\]
par ailleurs, $A$ étant borné, la série $\sum_j \lambda(R_j)$ est
convergente. Il existe donc un entier $n_0$ tel que $\sum_{j > n_0}
\lambda(R_j) < \frac{\epsilon}{2}$. L'application~:
\[
g = \sum_{j=1}^{n_0} 1_{R_j} 
\]
vérifie alors $\| 1_A -g \|_1 < \epsilon$. 
Soit maintenant un hyperrectangle $R = \prod_{i=1\dots n} ]a_i, b_i[$
  et soit pour $\eta > 0$, l'application~:
\[
h_\eta : x \to \left  \{
\begin{array}{ll}
0 & x \notin [0,1] \\
\frac{x}{\eta} & 0 \leq x < \eta \\
1 & \eta \leq x < 1-\eta \\
\frac{1-x}{\eta} & 1-\eta \leq x \leq 1
\end{array}
\right .
\]
L'application~:
\[
g : x \to \prod_{i=1\dots n} h_\eta \left ( \frac{x_i - a_i +\eta}{b_i
  - a_i + 2 \eta} 
\right )
\]
est continue et~:
\[
\| 1_R - g \|_1 \leq \prod_{i=1\dots n} (b_i - a_i + 2 \eta) -
\prod_{i=1\dots n} (b_i - a_i) 
\]  
ce qui achève la démonstration, $\eta$ étant arbitraire.
\end{proof}
\begin{prop}\label{dens:1}
L'ensemble des applications indéfiniment dérivables à support compact
est dense dans $L^1(\mathbb{R}^n)$. 
\end{prop}
\begin{proof}
Il suffit de trouver une application indéfiniment dérivable vérifiant
les mêmes propriétés que $h$ dans la démonstration précédente. Il
existe tout d'abord des applications indéfiniment dérivables à support
compact. On peut par exemple prendre, sur $[0,1]$~:
\[
\psi : x \to \exp \left (
\frac{-1}{x(1-x)}
\right )
\]
Considérons maintenant~:
\[
\phi : x \to \left  \{
\begin{array}{ll}
0 & x < 0 \\
\int_{[0,x]} \psi d \lambda / \int_{[0,1]} \psi d \lambda & x \in ]0,
1[ \\
1 & x \geq 1
\end{array}
\right .
\]
L'application $\psi$ est indéfiniment dérivable. On est maintenant en
mesure d'exhiber une application semblable à $h$ de la démonstration
précédente, mais indéfiniment dérivable~:
\[
h_\eta : x \to \left  \{
\begin{array}{ll}
0 & x \notin [0,1] \\
\phi \left (\frac{x}{\eta} \right ) & 0 \geq x < \eta \\
1 & \eta \geq x < 1-\eta \\
\phi \left (\frac{1-x}{\eta} \right ) & 1-\eta \geq x \leq 1
\end{array}
\right .
\]
\end{proof}
\begin{mandatory}
\begin{theorem} (Riemann-Lebesgue)
Soit $f \in L^1(\mathbb{R}^n)$. On a~:
\[
\lim_{\| \xi \| \to \infty} |\widehat{f}(\xi)| = 0
\]
\end{theorem}
\end{mandatory}
\begin{proof}
Soit $\phi$ application indéfiniment dérivable à support compact. On
a~:
\[
(1+\| \xi \|^2) \widehat{\phi}(\xi) = \mathcal{F}\left(\phi + \sum_{i=1}^n
  D^2_i \phi \right)(\xi) \leq \| \phi + \sum_{i=1}^n
  D^2_i \phi \|_1
\]
Le résultat est immédiat dans ce cas. Pour $f \in L^1(\mathbb{R}^n)$
quelconque, on applique la densité des applications indéfiniment
dérivables à support compact pour montrer que pour tout $\epsilon > 0$
il existe $\phi$ indéfiniment dérivable à support compact vérifiant
$\| f - \phi \|_1 < \epsilon$. Comme par ailleurs~:
\[
\widehat{f}(\xi) \leq \| f- \phi \|_1 + \widehat{\phi}(\xi)
\]
on a le résultat général.
\end{proof}
\begin{mandatory}
\begin{prop}
Soit $f \in L^1(\mathbb{R}^n)$ admettant une dérivée partielle $D_j f$
elle-même dans $ L^1(\mathbb{R}^n)$. On a~:
\[
\widehat{D_j f}(\xi) =  i \xi_j \widehat{f}(\xi)
\]
avec $\xi_j$ $j$-ième composante du vecteur $\xi$.
\end{prop}
\end{mandatory}
\begin{proof}
Il suffit de montrer le théorème pour une application $f : \mathbb{R}
\to \mathbb{C}$.
On a~:
\[
\widehat{f^\prime}(\xi) = \int_{\mathbb{R}} f^\prime (x) e^{-i \langle \xi,
  x \rangle } d \lambda(x)
\]
Le théorème de convergence dominée permet d'écrire~:
\[
\lim_{k \to + \infty} \int_{[-k,k]} f^\prime (x) e^{-i \langle \xi,
  x \rangle} d \lambda(x) = \widehat{f\prime}(\xi)
\]
en faisant une intégration par parties, on obtient~:
\begin{align*}
& \int_{[-k,k]} f^\prime(x) e^{-i \langle \xi,
  x \rangle } d \lambda(x) = \\
&[f(x) e^{-i \langle
  x, \xi \rangle }]_{-k}^k +  \int_{[-k,k]} i \xi f(x) e^{-i \langle\xi,
  x \rangle } d \lambda(x)
\end{align*}
comme par ailleurs $f(x) = f(0) +\int_{[0,x]} f^\prime(u) du$, $f$
admet une limite en $+\infty, -\infty$.$f$ étant sommable, cette
limite est nulle et l'on déduit le résultat en faisant $k \to +\infty$.
\end{proof}
Il est possible dans certains cas d'inverser la transformation de Fourier pour
obtenir la fonction de départ (en fait de façon plus précise, un représentant
de la même classe d'équivalence dans $L^1(\mathbb{R}^n$). Les contraintes qui
sont imposées à l'application dont on souhaite inverser la transformée de
Fourier sont néanmoins assez fortes (égalité presque partout avec une
application continue). Pour énoncer et démontrer la formule d'inversion, deux
lemmes sont nécessaires.
\begin{lemme}\label{fourier:3}
Soit l'application $\phi \colon \mathbb{R} \to \mathbb{R}$ définie par~:
\[
\phi : x \to \frac{1}{\sqrt{2\pi}}e^{-\frac{x^2}{2}}
\]
on a~:
\[
\widehat{\phi} : \xi \to e^{- \frac{\xi^2}{2}}
\]
\end{lemme}
\begin{proof}
L'application $\phi$ appartient à $L^1(\mathbb{R})$ et vérifie l'équation
différentielle $\phi^\prime + x \phi = 0$. Il suffit de montrer que $\widehat{\phi}$ vérifie la même.
Le théorème de dérivation des intégrales dépendant d'un paramètre
s'appliquant de façon évidente ($\phi$ est à décroissance
exponentielle)~:
\[
\widehat{\phi}^\prime(\xi) = -i \int
\frac{1}{\sqrt{2\pi}}e^{-\frac{x^2}{2}} x e^{-i \xi x} dx
\]
Par ailleurs, $\lim_{|x| \to +\infty} \phi(x) = 0$, d'où par intégration
par parties~:
\[
\widehat{\phi}^\prime(\xi) = - \xi \widehat{\phi}
\]
on en déduit que $\widehat{\phi}$ est solution de l'équation différentielle:
$$\widehat{\phi}^\prime(\xi)+\xi \widehat{\phi}(\xi)=0$$
 et donc:
\[
\widehat{\phi}(\xi) = K e^{-\frac{\xi^2}{2}}
\]
La constante $K$ se détermine en remarquant que l'on doit avoir~:
\[
\widehat{\phi}(0) = \int \frac{1}{\sqrt{2\pi}}e^{-\frac{x^2}{2}} dx = 1
\]
\end{proof}
Dans la suite, on posera pour $a > 0$:
\[
\phi_a \colon x \in \mathbb{R} \mapsto \frac{1}{a}
\exp\left(-\frac{x^2}{2a^2}\right)
\]
\begin{lemme}\label{fourier:4}
Soit $f \in L^1(\mathbb{R})$. On a $\lim_{a \to 0^+} f *
\phi_a = f$ dans $L^1(\mathbb{R})$.
\end{lemme}
\begin{proof}
On suppose dans un premier temps que $f$ est continue à support compact.
On a, en utilisant la parité de $\phi_a$:
\begin{align*}
f * \phi_a(y) & = \int_{\mathbb{R}} f(x) \phi_a(y-x) d \lambda(x) \\
&= \frac{1}{\sqrt{2\pi}} \int_\mathbb{R} f(x) a^{-1}
\exp\left(-\frac{(x-y)^2}{2a^2}\right) d \lambda(x)
\end{align*}
En effectuant le changement de variable $u=(x-y)a^{-1}$, on en déduit:
\[
f * \phi_a(y) = \frac{1}{\sqrt{2\pi}} \int_{\mathbb{R}} f(au+y)
\exp\left(-\frac{u^2}{2}\right) d\lambda(u)
\]
$f$ continue à support compact est bornée en valeur absolue par une constante
$M$, on a donc pour tout $u \in \mathbb{R}$:
\[
\left|f(au+y)
\exp\left(-\frac{u^2}{2}\right)\right| \leq M \exp\left(-\frac{u^2}{2}\right)
\]
Le théorème de convergence dominée s'applique, et en utilisant la continuité de
$f$, on a finalement pour $a \to 0^+$:
\[
\lim_{a \to 0^+} f * \phi_a(y) = f(y)
\]
Soit maintenant $f\in L^1(\mathbb{R})$ quelconque. Il existe une suite
d'applications continues à support compact $(f_n)_{n \in \mathbb{N}}$ de limite
$f$ dans $L^1(\mathbb{R})$. On a:
\[
\|f*\phi_a - f_n*\phi_a\|_1 = \|(f-f-n)*\phi_a\|_1 \leq \|f-f_n\|_1\|\phi_a\|_1
= \|f-f_n\|_1
\]
En passant à la limite $a \to 0^+$, on en déduit:
\[
\|\lim_{a\to 0^+}f*\phi_a - f_n\|_1  \leq \|f-f_n\|_1
\]
Puis, en faisant $n \to +\infty$:
\[
\|\lim_{a\to 0^+}f*\phi_a - f\|_1 = 0
\]
qui donne le résultat annoncé
\end{proof}
Ce résultat est intéressant en soi: il montre l'existence d'une approximation de
l'unité dans $L^1(\mathbb{R})$ pour le produit de convolution. On peut
utiliser cette propriété pour adjoindre une unité $\delta$ à $L^1(\mathbb{R})$
(mais cette unité n'est pas elle-même une application de $L^1(\mathbb{R})$). 

La formule d'inversion permet de calculer une application dont la tansformée
de Fourier est connue. Bien que relativement restrictive, elle est fréquemment
utilisée en traitement du signal pour passer du spectre fréquentiel au signal
qui lui a donné naissance.
\begin{mandatory}
\begin{theorem}{(Formule d'inversion)}
Soit $f \in L^1(\mathbb{R}^n)$ telle que $\widehat{f} \in L^1(\mathbb{R}^n)$.
L'application:
\[
x \in \mathbb{R}^n \mapsto \frac{1}{(2\pi)^n} \int_{\mathbb{R}^n}
\widehat{f}(\xi)\exp\left(i \langle x, \xi \rangle\right) d \lambda(\xi)
\]
est égale presque partout à $f$.
\end{theorem}
\end{mandatory}
\begin{proof}
Pour simplifier la preuve, on se placera dans $L^1(\mathbb{R})$, l'extension au
cas multi-dimensionnel se faisant par application du théorème de Fubini.
Soit $f \in L^1(\mathbb{R})$ telle que $\widehat{f} \in L^1(\mathbb{R})$. Pour
tout $a>0$, la proposition \ref{fourier:1} et le lemme \ref{fourier:3}
permettent d'écrire:
\begin{align*}
& \int_{\mathbb{R}}f(x) \frac{1}{a}\exp\left(-\frac{(y-x)^2}{2a^2}\right) d
\lambda(x) = \\
& \frac{1}{\sqrt{2\pi}} \int_{\mathbb{R}}
\widehat{f}(\xi)\exp\left(i \langle x, \xi
\rangle\right)\exp\left(-\frac{a^2\xi^2}{2}\right) d \lambda(\xi)
\end{align*}
Le membre de gauche de l'égalité est $\sqrt{2\pi}f*\phi_a$. 
On a de plus:
\[
\forall \xi \in \mathbb{R}, \,
\left|\widehat{f}(\xi)\exp\left(i \langle x, \xi
\rangle\right)\exp\left(-\frac{a^2\xi^2}{2}\right)\right| \leq |\widehat{f}(\xi)|
\]
Le théorème de convergence dominée s'applique donc au membre de droite et on en
déduit:
\[
\lim_{a \to 0^+} f * \phi_a = \frac{1}{2 \pi}  \int_{\mathbb{R}}
\widehat{f}(\xi)\exp\left(i \langle x, \xi
\rangle\right) d \lambda(\xi)
\]
Le lemme \ref{fourier:4} donne la conclusion recherchée.
\end{proof}
\begin{rem}
La transformée de Fourier inverse, au même titre que la transformée directe, est
une application bornée continue. Une application de $L^1(\mathbb{R}^n)$ ne
pourra donc avoir une transformée de Fourier dans $L^1(\mathbb{R}^n)$
que si elle est égale presque partout à une application continue bornée. 
\end{rem}
La transformation de Fourier possède de bonnes propriétés vis-à-vis du produit
de convolution:
\begin{mandatory}
\begin{prop}
Soient $f,g \in L^1(\mathbb{R})$. On a:
$$\mathcal{F}(f*g)=\mathcal{F}(f).\mathcal{F}(g)$$
 De plus, si
$\widehat{f},\widehat{g}$ sont dans $L^1(\mathbb{R})$, on a:
$$\mathcal{F}^{-1}(\widehat{f}*\widehat{g})=2 \pi
\mathcal{F}^{-1}(\widehat{f}).\mathcal{F}^{-1}(\widehat{g})=2 \pi f . g$$
\end{prop}
\end{mandatory}
\begin{proof}
Par définition du produit de convolution:
\[
(f*g)(y) = \int_{\mathbb{R}} f(y-x)g(x) d\lambda(x)
\]
Sa transformée de Fourier est:
\[
\mathcal{F}(f*g) \colon \xi \mapsto \int_{\mathbb{R}} \left ( \int_{\mathbb{R}}
f(y-x)g(x) d\lambda(x) \right) e^{-i y \xi} d\lambda(y)
\]
L'application:
\[
(x,y) \in \mathbb{R}^2 \mapsto f(y-x)g(x)e^{-iy \xi}
\]
est sommable (cf démonstration dans la section sur le produit de convolution),
le théorème de Fubini et le changement de variable $u=y-x$ donnent:
\begin{align*}
\mathcal{F}(f*g)(\xi) & = \int_{\mathbb{R}^2}
f(y-x)g(x)   e^{-i (y-x) \xi}e^{-i x \xi} d\lambda(x)d\lambda(y) \\
& = \int_{\mathbb{R}^2}
f(u)g(x)   e^{-i u \xi}e^{-i x \xi} d\lambda(x)d\lambda(u) \\
& \int_{\mathbb{R}}
f(u)   e^{-i u \xi}d\lambda(u)  \int_{\mathbb{R}}g(x) e^{-i x \xi}
d\lambda(x) = \mathcal{F}(f)(\xi).\mathcal{F}(g)(\xi)
\end{align*}
La propostion relative à la transformée inverse est en tout point similaire, une
précaution étant néanmoins à prendre du fait de la présence du facteur $2\pi$.
\end{proof}
\section{Transformation de Fourier dans $L^2$}
La transformation de Fourier des applications de  $L^2(\mathbb{R}^n)$ ne peut pas se faire directement
à l'aide d'intégrales. Il faut procéder par extension en définissant
dans un premier temps cette transformation sur une classe plus
restreinte d'applications, stable par transformation de Fourier et dense dans 
$L^2(\mathbb{R}^n)$. Cette classe est celle des applications indéfiniment
dérivables à décroissance rapide que l'on va maintenant introduire, et sur
laqulle on montrera que la transformation de Fourier est un endomorphisme.
\begin{mandatory} 
\begin{defn}
Une application $f : \mathbb{R}^n \to \mathbb{C}$ est dite
indéfiniment dérivable à décroissance rapide si~:
\begin{itemize}
\item $f \in C^\infty(\mathbb{R}^n)$
\item $\forall (n,m) \in
\mathbb{N}^2, \, \lim_{|x|\to +\infty} \left |
x^n \frac{d^m f}{dx^m} \right | = 0$
\end{itemize}
\end{defn}
\end{mandatory}
\begin{term}
L'ensemble des applications indéfiniment dérivables à décroissance
rapide est appelé classe des applications de Schwartz et se note $\mathcal{S}( \mathbb{R}^n)$.
\end{term}

Il est clair que toute application de la classe $\mathcal{S}$ est
élément de $L^p(\mathbb{R}^n)$ pour tout $p \geq 1$. On définira la
transformée de Fourier d'une application de $\mathcal{S}$ par la
formule intégrale vue pour les applications sommables. 
\begin{mandatory}
\begin{prop}
Soit $f \in \mathcal{S}$. La transformée de Fourier de $f$ est encore
une application de $\mathcal{S}$.
\end{prop}
\end{mandatory}
\begin{proof}
Le caractère indéfiniment dérivable s'obtient aisément à partir du
théorème de dérivation sous le signe somme. Soit en effet un
multi-indice $I$ de longueur $p$. On écrit, pour tout $x$~:
\[
D_I e^{- i \langle x, \xi \rangle} = (-i)^n \left (\prod_{i \in I} x_i
\right )  e^{- i \langle x, \xi \rangle}
\]
Le terme $\prod_{i \in I} x_i$ se majore en valeur absolue par
$(1+\|x\|^2)^p$. Comme par ailleurs il est clair que les applications
$f(x)(1+\|x\|^2)^N$ sont sommables pour tout $N$, la dérivablité
d'ordre quelconque est obtenue.
En ce qui concerne la décroissance rapide, il suffit de reprendre la
démonstration du théorème de Riemann-Lebesgue en poursuivant les
intégrations par parties aussi loin que nécessaire (le caratère
$C^\infty$ des applications de $\mathcal{S}$ autorisant cette opération).
\end{proof}
Les applications de la classe $\mathcal{S}$ étant $L^1(\mathbb{R}^n)$, la
formule d'inversion précédemment établie est valable: la transformation de
Fourier est inversible en tant qu'opérateur linéaire continu $\mathcal{F}$ de
$\mathcal{S}$ dans lui-même et~:
\[
\mathcal{F}^{-1}(\widehat{f},x) = (2 \pi)^{-n} \int_{\mathbb{R}^n}
\widehat{f}(\xi) e^{i\langle \xi, y\rangle} d \lambda(\xi)
\]
\begin{mandatory}
\begin{prop}
Soit $f \in \mathcal{S}$ et $\widehat{f}$ sa transformée de
Fourier. On a~:
\[
 \int \|f\|^2 d \lambda = (2\pi)^{-n} \int \|\widehat{f}\|^2 d \lambda
\]
\end{prop}
\end{mandatory}
\begin{proof}
On vérifie immédiatement que pour tout $f \in \mathcal{S}$~:
\[
\overline{\mathcal{F}^{-1}(\widehat{f})} = \mathcal{F}(\overline{\widehat{f}})
\]
La proposition annoncée est alors un cas particulier de la proposition \ref{fourier:1}
\end{proof}
Cette dernière proposition montre que la transformation de fourier est
un endomorphisme quasi-isométrique de $\mathcal{S}$.
La définition de la transformation de Fourier des applications de
$L^2$ va se faire en utilisant la densité des applications
de $\mathcal{S}$ dans $L^2$ (voir proposition \ref{dens:1})
\begin{prop}
Soit $E$ un espace métrique et $A$ une partie dense de $E$. Soit $f
: A \to \mathbb{C}$ uniformément continue. Il existe une unique
extension continue de $f$ à $E$ que l'on continuera à noter $f$ par
abus de langage.
\end{prop}
\begin{proof}
Pour tout $x \in E$, il existe une suite $(x_n)$ d'éléments de $A$
avec $x = \lim_n x_n$. On définit l'extension par $f(x) = \lim_n
f(x_n)$. La suite $x_n$ étant convergente et $f$ continue, on vérifie
que la suite $f(x_n)$ est de Cauchy dans $\mathbb{C}$ complet, donc
admet une limite. Montrons que cette extension est continue en tout
point $x \in E$. Soit $\epsilon > 0$.  $f$ étant uniformément
continue, il existe  $\eta$ tel que~:
\[
\forall (u,v) \in A^2, d(u,v) < \eta
\Rightarrow |f(u)-f(v)| < \frac{\epsilon}{3}
\]
Par ailleurs, soit $(x_n)$ suite d'éléments de $A$ de limite $x$ et
$(y_n)$ de limite $y$. Il existe $n_0$ tel que pour tout $n \geq n_0$,
$d(x_n,x) < \frac{\eta}{3}$, $d(y_n,y) < \frac{\eta}{3}$ et
$|f(x_n)-f(x)| < \frac{\epsilon}{3}$, $|f(y_n)-f(y)| <
\frac{\epsilon}{3}$. On obtient alors $|f(x)-f(y)|< \epsilon$ pour
$d(x,y) < \frac{\eta}{3}$.
\end{proof}
\begin{mandatory}
\begin{defn}
La transformée de Fourier est l'unique extension à $L^2$ de
l'endomorphisme quasi-isométrique continu défini sur $\mathcal{S}$.
\end{defn}
\end{mandatory}
La transformation de Fourier ainsi définie est encore un endomorphisme
quasi-isométrique sur $L^2$ et pour tout $f \in L^2(\mathbb{R}^n)$:
\begin{mandatory}
\[
 \int \|f\|^2 d \lambda = (2\pi)^{-n} \int \|\widehat{f}\|^2 d \lambda
\]
\end{mandatory}
Les propriétés de la transformation de Fourier 
relativement aux translations, changements d'échelle et dérivations partielles
sont préservées dans $L^2(\mathbb{R}^n)$. Sous réserve du caractère $L^2$ des
produits, la proposition relative aux produits de convolution reste elle aussi
valable.
 Les définitions de la transformation de Fourier dans $L^1$ et dans
$L^2$ sont de formes très différentes. En revanche, il y a compatibilité au sens donné par la proposition suivante:
\begin{mandatory}
\begin{prop}
Soit $f \in L^1(\mathbb{R}^n) \cap L^2(\mathbb{R}^n)$. Alors
l'application:
\[
\widehat{f}(\xi) = \int_{\mathbb{R}^n} f(x) \exp \left(- i \langle x ,
\xi \rangle \right) d\lambda(x)
\]
est égale à la transformée de Fourier de $f$ dans $L^2(\mathbb{R}^n)$
\end{prop}
\end{mandatory}
\begin{proof}
Soit $f \in L^1(\mathbb{R}^n) \cap L^2(\mathbb{R}^n)$. On distinguera la
transformée de Fourier de $f$ dans $L^1(\mathbb{R}^n)$ de celle dans
$L^2(\mathbb{R}^n)$ en notant la première $\widehat{f}$ et la seconde
$\mathcal{F}(f)$, cette convention s'appliquant uniquement à la preuve courante.
Soit $\psi \in \mathcal{S}$ quelconque. La proposition \ref{fourier:1} donne:
\[
\int_{\mathbb{R}^n} \widehat{f}(\xi) \psi(\xi) d \lambda(\xi) =
\int_{\mathbb{R}^n} f(x) \widehat{\psi}(x) d \lambda(x)
\]
Soit $(\phi_n)_{n \in \mathbb{N}}$ une suite d'applications de $\mathcal{S}$
ayant pour limite $f$ dans $ L^2(\mathbb{R}^n)$. Par continuité du produit
scalaire (conséquence immédiate de l'inégalité de Cauchy-Schwartz):
\[
\lim_{n \to +\infty} \int_{\mathbb{R}^n} \phi_n(x) \widehat{\psi}(x) d
\lambda(x) = \int_{\mathbb{R}^n} f(x) \widehat{\psi}(x) d \lambda(x)
\]
La proposition \ref{fourier:1} donne:
\begin{align*}
& \lim_{n \to +\infty} \int_{\mathbb{R}^n} \phi_n(x) \widehat{\psi}(x) d
\lambda(x) = \\
  \lim_{n \to +\infty} \int_{\mathbb{R}^n} \widehat{\phi_n}(\xi)
  \psi(\xi) d \lambda(\xi) = \\
  & \int_{\mathbb{R}^n} \mathcal{F}(f)(\xi)
  \psi(\xi) d \lambda(\xi)
\end{align*}
On en déduit donc:
\[
\int_{\mathbb{R}^n} \left(\mathcal{F}(f)(\xi)-\widehat{f}(\xi)\right)
  \psi(\xi) d \lambda(\xi) = 0
\]
Puis par continuité du produit scalaire et densité de $\mathcal{S}$ dans 
$L^2(\mathbb{R}^n)$:
\[
\forall g \in  L^2(\mathbb{R}^n), \quad \int_{\mathbb{R}^n} \left(\mathcal{F}(f)(\xi)-\widehat{f}(\xi)\right)
  g(\xi) d \lambda(\xi) = 0
\]
prouvant ainsi que $\mathcal{F}(f)-\widehat{f}=0$ dans $L^2(\mathbb{R}^n)$.
\end{proof}

Le calcul pratique d'une transformation de Fourier dans $L^2(\mathbb{R}^n)$ est
parfois délicat: on ne peut généralement pas utiliser de forme intégrale, mais
uniquement des suites d'applications convergeant au sens $L^2$ vers
l'application cible. 
Une formule souvent utile est présentée dans l'exercice ci-dessous.
\begin{exercice}
Soit $f \in L^2(\mathbb{R})$. Soit $a > 0$. 
\begin{enumerate}
  \item Montrer que l'application $f_a = f 1_{[-a,a]}$ est sommable et que
  $\lim_{a \to +\infty}f_a=f$ dans $ L^2(\mathbb{R})$.
  \item En déduire que la transformée de Fourier de $f$ dans $L^2(\mathbb{R})$
  est la limite au sens $L^2$ pour $a \to +\infty$ de la famille d'applications:
  \[
  \xi \mapsto \int_{[-a,a]} f(x) \exp(-ix\xi) d\lambda(x)
  \]
\end{enumerate}
\end{exercice}
\begin{rem}
Cette formule n'apporte en soit rien de plus par rapport à la définition de la
transformée de Fourier utilisant les applications de la classe $\mathcal{S}$. En
revanche, si la limite:
\[
 \lim_{a \to +\infty} \int_{[-a,a]} f(x) \exp(-ix\xi) d\lambda(x)
\]
peut se calculer pour presque tout $\xi$, elle donne automatiquement un
représentant de la classe de la transformée de Fourier de $f$. Nous
verrons dans la suite du cours que des techniques utilisant des intégrales dans
le plan complexe permettent souvent de calculer des intégrales
généralisées de cette forme: beaucoup de transformations de Fourier sont
obtenues avec cette méthode.
\begin{exercice}
\begin{enumerate}
  \item Déterminer la transformation de Fourier inverse dans $L^2(\mathbb{R})$
  de l'application $1_{[-1,1]}$.
  \item En déduire la transformation de Fourier de l'application $\sinc$:
  \[
  x \in \mathbb{R} \mapsto \sinc(x) = \left\{
  \begin{array}{cc}
  	1 & \text{ si } x=0 \\
  	\frac{\sin x}{x} & \text{ si } x \neq 0 
  \end{array}
  \right.
  \]
  \item Pouvez-vous en déduire que $\sinc \notin L^1(\mathbb{R})$ ?
\end{enumerate}
\end{exercice}
Le caractère quasi-isométrique de la transformée de Fourier est parfois utile
pour calculer des intégrales lorsque la fonction à intégrer est le carré d'une
application $L^2$:
\begin{exercice}
Soit l'application~:
\[
f : x \to \left \{
\begin{array}{ll}
\frac{\sin^2 x }{x^2} & x \neq 0 \\
1 & x = 0
\end{array}
\right .
\]
\begin{itemize}
\item Montrer que $f \in L^1(\mathbb{R})$ et calculer $\widehat{f}$
(on pourra utiliser la transformation inverse de Fourier dans
$L^2(\mathbb{R})$ de l'application $\pi 1_{[-1,+1]}$ vue dans l'exercice
précédent et le produit de convolution).
\item En déduire la valeur des l'intégrales~:
\[
\int_{\mathbb{R}} \frac{\sin^2 x}{x^2} d \lambda(x)
\]
et
\[
\int_{\mathbb{R}} \frac{\sin^4 x}{x^4} d \lambda(x)
\]
\end{itemize}
\end{exercice}
\end{rem}
