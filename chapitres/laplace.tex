\chapter{Transformation de Laplace}
\section{Définition}
Soit $f : \mathbb{R} \to \mathbb{C}$ une application telle que~:
\begin{itemize}
\item $\forall t < 0, \, f(t) = 0$.
\item $f$ est intégrable sur tout compact de $\mathbb{R}$.
\item Il existe un réel $x$ tel que l'intégrale~:
\[
\int_{\mathbb{R}^+}e^{-xt} f(t) dt
\]
soit convergente.
\end{itemize}
\begin{mandatory}
\begin{defn}
Soit $f$ une application vérifiant les conditions précédentes. La
transformée de Laplace de $f$ est l'application $\mathcal{L}(f) : \Omega \subset \mathbb{C} \to
\mathbb{C}$ définie par~:
\[
\mathcal{L}(f)(p) = \int_0^{+\infty}e^{-pt} f(t) dt
\]
$\Omega$ étant l'ensemble des $p$ pour lesquels l'intégrale précédente
est définie.
\end{defn}
\end{mandatory}
\begin{mandatory}
\begin{prop}
Si la transformation de Laplace de $f$ est définie pour $p_0$, elle
est également définie pour tout $p$ tel que $\Re(p) > \Re(p_0)$.
\end{prop} 
\end{mandatory}
La borne inférieure de l'ensemble des $x$ pour lesquels l'intégrale de
Laplace est convergente pour $\Re(p) = x$ est appelée abscisse de
convergence de la transformation de Laplace.
On définira de même l'abscisse de convergence absolue, qui est
supérieure ou égale à l'abscisse de convergence. On remarquera que
chacune de ces bornes peut être $-\infty$.
\begin{mandatory}
\begin{prop}
Soit $p_0$ un complexe tel que la transformée de Laplace $\mathcal{L}(f)(p_0)$ de
$f$ soit définie. Soit $C_{\alpha, p_0}$ l'ensemble des complexes de
la forme $p_0 + r e^{i \theta}$ avec $r \in \mathbb{R}^+$ et $\theta
\in [-\alpha, \alpha]$. L'intégrale de Laplace est uniformément
convergente dans $C_{\alpha, p_0}$ si $\alpha \in ]0, \frac{\pi}{2}[$.
\end{prop}
\end{mandatory}
\begin{proof}
Soit $\epsilon > 0$ et soit $T > 0$ tel que pour tout $t > T$~:
\[
\int_T^t e^{-p_0 u}f(u) du < \epsilon
\]
On note $g(t) = \int_T^t e^{-p_0 u}f(u) du$.
Posons $p = p_0 + h$ pour $p \in C_{\alpha, p_0}$ et $h = r e^{i
\theta}$. On a~:
\[
\int_T^t e^{-p u}f(u) du = \int_T^t e^{-hu}g^\prime(u) du
\]
Une intégration par parties conduit à~:
\[
\int_T^t e^{-hu}g^\prime(u) dt = e^{-ht}g(t) + h \int_T^t e^{-hu}g(u) du
\]
Pour $t > T$ le premier terme est inférieur à $\epsilon$, alors que le
second est borné en module par~:
\[
r \epsilon \int_T^{+\infty} e^{-r u \cos(\alpha)} d u =
\frac{\epsilon}{\cos(\alpha)} 
\]
\end{proof}
La transformation de Laplace est holomorphe dans son domaine de
convergence comme le montre la proposition suivante.
\begin{mandatory}
\begin{prop}
Soit $f$ continue sur $\mathbb{R}^+$ et soit $s$ l'abscisse de
convergence de sa transforméee de Laplace $F$. Alors $F$ est
holomorphe dans le demi-plan complexe formé des points de partie
réelle strictement supérieure à $s$.
\end{prop}
\end{mandatory}
\begin{proof}
Soit $p = x+iy$ avec $x > s$. Soit $p_0 = x_0 + i y$ un point tel
que $s < x_0 < x$. Pour tout $\alpha \in ]0, \frac{\pi}{2}[$, le point
$p$ appartient à $C_{\alpha, p_0}$ et l'intégrale définissant $F$
est uniformément convergente dans $C_{\alpha, p_0}$. De plus,
l'application~:
\[
p \to e^{-pt} f(t)
\]
est indéfiniment dérivable sur $C_{\alpha, p_0}$ et $t > 0$. On en
déduit alors (dérivation sous le signe somme) que $F(p)$ est
holomorphe dans $C_{\alpha, p_0}$ et que~:
\[
F^{(n)}(p) = (-1)^n \int_0^{+\infty} t^n f(t) e^{-pt} dt
\] 
\end{proof}
\begin{rem}
La proposition reste vraie si $f$ est seulement continue par morceaux.
\end{rem}
\section{Propriétés}
La transformation de Laplace est de façon évidente linéaire. Son
intérêt principal vient de son comportement vis à vis de la
dérivation.
\begin{mandatory}
\begin{prop}
Soit $f$ continue sur $\mathbb{R}^+$, dérivable. Si elle admet une
transformée de laplcace ainsi que sa dérivée, les deux transformées
sont liées par la relation~:
\[
\mathcal{L}(f^\prime)(p) = p \mathcal{L}(f)(p) - f(0^+)
\] 
avec $f(0^+) = \lim_{ t \to 0, t > 0} f(t)$
\end{prop}
\end{mandatory}
\begin{proof}
On a~:
\[
\mathcal{L}(f^\prime)(p) = \lim_{t\to + \infty} \int_0^Tf^\prime(t) e^{-pt} dt
\]
Une intégration par parties montre que l'intégrale de droite vaut~:
\[
\left [
e^{-pt}f(t)
\right ]_0^T
+ p \int_0^T f(t) e^{-pt} dt
\]
Qui montre la proposition par passage à la limite.
\end{proof}
Cette proposition s'étend par récurrence pour les dérivées d'ordre
quelconque~:
\[
\mathcal{L}(f^{(n)})(p) = p^n \mathcal{L}(f)(p) - p^{n-1} f(0)
- p^{n-2} f^\prime(0) - \dots f^{(n-1)}(O^+))
\]
Il est possible de déterminer également la transformée de Laplace de
la primitive d'une application.
\begin{mandatory}
\begin{prop}
Si $f$ admet une transformée de Laplace $F$ et si $\lim_{t \to 0^+}
\frac{f(t)}{t}$ existe, alors~:
\[
\mathcal{L}(f(t)/t)(p) = \int_p^{+\infty} F(u) du
\]
\end{prop} 
\end{mandatory}
\begin{proof}
\[
\int_p^Z F(u) du = \int_p^Z \int_0^{+\infty} e^{-ut}
f(t) dt du
\]
comme l'intégrale de Laplace est uniformément convergente dans un
secteur angulaire, on peut intervertir l'ordre de intégrations et
l'intégrale devient~:
\[
\int_0^{+\infty} e^{-pt} \frac{f(t)}{t} dt - \int_0^{+\infty} e^{-Zt} \frac{f(t)}{t} dt
\] 
La conclusion s'obtient en faisant tendre $Z$ vers l'infini en module.
\end{proof}
\begin{mandatory}
\begin{prop}
Soit $f$ de transformée de Laplace $F$ et soit $s$ l'abscisse de
convergence de cette transformée. Alors l'application~:
\[
\int_0^t f(u) du
\]
admet pour transformée de Laplace $\frac{F(p)}{p}$ et son abscisse de
convergence est strictement supérieure à $\sup(0, s)$.
\end{prop}
\end{mandatory}
Cette proposition se montre de façon similaire à la précédente.
\begin{prop}
Soit $f$ admettant une transformée de Laplace $F$ d'abscisse de
convergence $s$. Soit $a \in \mathbb{C}$. Alors l'application
$e^{at}f(t)$ admet pour transformée de Laplace $F(p-a)$, son abscisse
de convergence étant $s + \Re(a)$.
\end{prop}
\begin{prop}
Soit $f$ admettant $F$ pour transformée de Laplace. Soit $a \in
\mathbb{R}$. Alors
l'application $g$ telle que $g(t) = 0$ si $t < a$ et $g(t) = f(t-a)$
sinon admet pour transforméee de Laplace $e^{-ap}F(p)$.
\end{prop}
Ces deux propositions se montrent de manière immédiate.
On peut calculer facilement à partir de ceci la transformée de Laplace
d'une application périodique. Soit $f_1$ une application à support
dans un intervalle réel $[0,T]$ et soit $f$ l'application définie par
périodisation de $f_1$ de période $T$. Si $f$ admet une transformée de
Laplace et si $F_1$ est la transformée de Laplace de $f_1$, alors~:
\[
\mathcal{L}(f)(p) = \frac{F_1(p)}{1-e^{-pT}}
\]
Ce résultat s'obtient en écrivant $f$ sous la forme d'une somme de
termes de la forme $f_1(t-k T)$ avec $k \in \mathbb{N}$ et en faisant
apparaître une série géométrique.
On a enfin une proposition relative aux changements d'échelle.
\begin{prop}
Si $f$ admet pour transformée de Laplace $F$ alors, pour $\lambda \in
\mathbb{R^*}$, l'application $f(t/\lambda)$ admet pour transformée de
Laplace $\lambda F(\lambda p)$
\end{prop}
\section{Inversion}
\begin{mandatory}
\begin{prop}(Théorème de la valeur finale)
Soit $f$ continue de transformée de Laplace $F$. Si l'abscisse de convergence
$s$ est négative, alors~:
\[
\lim_{p \to O^+} p F(p) = \lim_{t \to +\infty} f(t)
\]
\end{prop}
\end{mandatory}
\begin{proof}
Soit $l = \lim_{t \to +\infty} f(t)$. 
\begin{align*}
pF(p) - l & = p \int_0^{+\infty} e^{-pt}f(t) dt - l \\
&= p \int_0^T  e^{-pt}f(t) dt + p  \int_T^{+\infty}e^{-pt}(f(t) -l) dt
\\ &+ l \left (
p \int_T^{+\infty} e^{-pt}dt -1
\right )
\end{align*}
Le premier terme est majoré en module par~:
\[
|p| T M_T
\]
avec $M_T$ borne sup de $|f|$ sur $[0,T]$.
Le second terme se majore uniformément par définition de la limite de
$f$ en $+\infty$ et la convergence uniforme de l'intégrale. Enfin, le
troisième terme tend vers 0 pour $|p|\to O^+|$, ce qui montre la proposition.
\end{proof}
\begin{mandatory}
\begin{prop}(Théorème de la valeur initiale)
Soit $f$ dérivable, de transforméee de Laplace $F$ et telle que sa
dérivée soit également transformable. Alors~:
\[
\lim_{|p|\to +\infty} p F(p) = f(0^+)
\]
\end{prop}
\end{mandatory}
\begin{proof}
On a~:
\[
\lim_{|p| \to +\infty} \mathcal{L}(f^\prime)(p) = 0
\]
et $\mathcal{L}(f^\prime)(p) = p F(p) - f(0^+)$.
\end{proof}
\begin{mandatory}
\begin{prop}
Soit $f$ définie par une série entiére convergente $\sum_{n \in
\mathbb{N}} a_n t^n / n!$ de rayon de convergence infini. Alors elle
est transformable et sa transformée s'obtient en transformant terme à terme.
\end{prop}
\end{mandatory}
\begin{defn}
Soit $[a,b]$ un intervalle de $\mathbb{R}$. Soit $f : [a,b] \to \mathbb{R} (\mbox{ resp. } \mathbb{C})$. On dira
que $f$ est à variation bornée si~:
\[
\sup \{ \sum_{i=0\dots N-1} |f(t_{i+1}) - f(t_i)|, \, a = t_0 < t_1, \dots < t_N = b, \, N \in \mathbb{N} \} < + \infty
\]
\end{defn}
La définition précédente s'étend immédiatement au cas des intervalles semi-infinis ou infinis.
La proposition suivant est trés classique~:
\begin{prop}(Lemme de Dirichlet)
Soit $f$ à variation bornée sur un intervalle $[0,a]$. On a~:
\[
\lim_{b \to +\infty} \int_[0,a] f(x) \frac{\sin bx}{x} d \lambda(x) =
f(0^+) \lim_{T \to +\infty} \int_[0,T] \frac{\sin x}{x} d \lambda(x)  
\]
\end{prop}
\begin{mandatory}
\begin{theorem}
Soit $F$ la transformée de Laplace d'une application à variation bornée $f$, l'abscisse
de convergence étant $s$. En tout point de continuité $t$ de $f$, on a~:
\[
f(t) = \frac{1}{i 2 \pi} \lim_{b \to + \infty} \int_{a-ib}^{a+ib}
e^{pt} F(p) dp
\]
avec $a > s$.
\end{theorem}
\end{mandatory}
\begin{proof}
Formons~:
\[
f_b(t) = \frac{1}{i 2 \pi}  \int_{a-ib}^{a+ib} e^{pt} \int_0^{\infty}
e^{-pu} f(u) du dp
\]
La convergence de l'intégrale étant uniforme, on peut écrire, après
interversion des intégrales~:
\[
f_b(t) = \frac{1}{\pi} e^{at} \int_{-t}^{+\infty} f(u+t)e^{-a(u+t)}
\frac{sin b u}{u} du
\]
Comme par ailleurs $f(t) = 0, \, t < 0$, on a~:
\[
f_b(t) = \frac{1}{\pi} e^{at} \int_{-\infty}^{+\infty} f(u+t)e^{-a(u+t)}
\frac{sin b u}{u} du
\]
Soit encore~:
\[
f_b(t) = \frac{1}{\pi} e^{at} \int_{0}^{+\infty} \left (
f(u+t)e^{-a(u+t)} + f(t-u)e^{-a(t-u)}
\right )
\frac{sin b u}{u} du
\]
En utilisant~:
\[
\int_0^{+\infty} \frac{sin b u}{u} du = \frac{\pi}{2}
\]
On obtient~:
\[
f_b(t) = f(t)  + \frac{1}{\pi} e^{at} \int_{0}^{+\infty} \left (
f(u+t)e^{-a(u+t)} + f(t-u)e^{-a(t-u)} - 2 f(t)e^{-at}
\right )
\frac{sin b u}{u} du
\]
Le résultat s'obtient alors en faisant tendre $b$ vers $+\infty$ et en utilisant le lemme de Dirichlet.
\end{proof} 
Il existe une forme réciproque de ce théorème~:
\begin{mandatory}
\begin{theorem}
Soit une application $F$ analytique dans un demi-plan $\Re(p) > s_0$,
tendant vers 0 pour $|p| \to +\infty$ dans tout demi-plan $\Re(p) > s
> s_0$. Si l'intégrale~:
\[
\int_{s-i \infty}^{s+i \infty} F(p) dp
\]
converge absolument pour $s > s_0$, alors $F$ est la transformée de
Laplace de l'application $f$ définie par~:
\[
f(t) = \frac{1}{i 2 \pi} \int_{s-i \infty}^{s+i \infty} F(p) dp
\]
\end{theorem}
\end{mandatory}
\begin{exercice}
Déterminer l'original de l'application~:
\[
F : p \to \frac{1}{1+p^3}
\]
en utilisant~:
\begin{itemize}
\item Les originaux élémentaires.
\item La formule d'inversion de Mellin-Fourier.
\end{itemize}
\end{exercice}