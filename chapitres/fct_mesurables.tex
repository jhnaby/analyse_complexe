\chapter{Fonctions mesurables}
\section{définitions}
Dans l'optique de la définition de l'intégrale en utilisant le procédé de
Lebesgue mentionné en introduction, il est nécessaire de caractériser les
applications pour lesquelles l'image inverse d'un intervalle est une partie
mesurable (i.e. appartient à la tribu de Borel ou à sa complétée dans le cas usuel). La
notion correspondante, replacée dans le cadre plus large des tribus, est
celle d'application mesurable.
\begin{mandatory}
\begin{defn}
Soient $(E, \mathcal{T})$, $(F, \mathcal{F})$ deux espaces mesurables. Soit $f : E \to F$. On dira que $f$ est mesurable si 
pour tout $A \in \mathcal{F}$, $f^{-1}(A) \in \mathcal{T}$.
\end{defn}
\end{mandatory}
Cette définition peut aussi s'interpréter en considérant les propriétés des
images réciproques:
\begin{prop}
Soient $(E, \mathcal{T})$, $(F, \mathcal{F})$ deux espaces mesurables et $f :
E \to F$ une application. $f^{-1}\left( \mathcal{F} \right)$ est une
tribu sur $E$
\end{prop}
\begin{proof}
De façon évidente, $\emptyset = f^{-1}(\emptyset)$, $E=f^{-1}(F)$. Soit $A \in
\mathcal{F}$. On a:
\begin{align*}
f^{-1}(A^c) & = \left \{ x \in E, \, f(x) \notin A \right \} \\
& = \left \{ x \in E, \, f(x) \in A \right \}^c \\
& = f^{-1}(A)^c
\end{align*}
On vérifie de m\^eme que pour toute famille dénombrable $(A_n)_{n \in
\mathbb{N}}$ d'élements de $\mathcal{F}$:
\[
f^{-1}\left(\bigcup_{n \in \mathbb{N}} A_n\right) = \bigcup_{n \in \mathbb{N}}
f^{-1}\left(A_n\right), \quad f^{-1}\left(\bigcap_{n \in \mathbb{N}} A_n\right)
=
\bigcap_{n \in \mathbb{N}} f^{-1}\left(A_n\right)\]
prouvant que l'image réciproque commute avec les opérations pour lesquelles une
tribu doit \^etre stable. 
\end{proof}
Le transport de la structure de tribu par une application n'est vrai que pour
l'image réciproque (faire le parallèle avec les applications continues entre
espaces topologiques). Dans le cas de l'image directe, certaines propriétés de
stabilité ne sont pas vérifiées. On a en particulier:
 \[
 f\left(\bigcap_{n \in \mathbb{N}} A_n\right) \subset \bigcap_{n \in
 \mathbb{N}} f\left(A_n\right)
\]
mais non l'égalité.
On voit que dans le cas d'une application $f$ mesurable, la tribu
$f^{-1}\left(\mathcal{F}\right)$ est incluse dans la tribu $\mathcal{T}$.
\begin{prop}
Soit $f \colon E \to F$ une application et $B \subset \mathcal{P}(E)$ un
ensemble de parties de $F$. On a:
\[
\mathcal{T}\left(f^{-1}(B)\right) = f^{-1}\left(\mathcal{T}\left(B\right)\right)
\]
\end{prop}
En d'autres termes, la tribu engendrée commute avec l'image réciproque.
\begin{proof}
On sait que $f^{-1}\left(\mathcal{T}\left(B\right)\right)$ est une tribu en
vertu de la proposition précédente. Comme elle contient les images réciproques
des éléments de $B$, on en déduit:
\[
\mathcal{T}\left(f^{-1}(B)\right) \subset
f^{-1}\left(\mathcal{T}\left(B\right)\right)
\]
Posons:
\[
\mathcal{G}= \left\{ A \in F, \, f^{-1}(A) \in \mathcal{T}\left(
f^{-1}(B)\right)\right\}
\]
On vérifie facilement que $\mathcal{G}$ est une tribu sur $F$, qui contient $B$
et donc qui contient $\mathcal{T}\left(B\right)$. On en déduit l'inclusion
réciproque:
\[
\mathcal{T}\left(f^{-1}(B)\right) \supset
f^{-1}\left(\mathcal{T}\left(B\right)\right)
\]
\end{proof}
\begin{mandatory}
\begin{corollaire}(critère de mesurabilité)
Soient $(E, \mathcal{T})$, $(F, \mathcal{F})$ deux espaces mesurables et $f :
E \to F$ une application. Soit $B$ un ensemble de parties de $F$ engendrant la
tribu $\mathcal{F}$. Pour que $f$ soit mesurable, il faut et il suffit que:
\[
\forall A \in B, \, f^{-1}(A) \in \mathcal{T}
\]
\end{corollaire}
\end{mandatory}
La preuve est une application directe de la proposition qui précéde.
Le caractère mesurable est lié intrinséquement aux tribus de départ et
d'arrivée: une application entre deux mêmes ensembles peut être mesurable pour
un choix particulier de tribus et non pour un autre. Il arrive parfois que l'on
recherche la plus grande tribu sur l'ensemble d'arrivée ou la plus petite tribu
sur l'ensemble de départ pour laquelle une application donnée est mesurable. La
contruction d'une tribu induite sur une partie d'un ensemble correspond à ce
cas:
\begin{mandatory}
\begin{prop}
Soit $(E, \mathcal{T})$ un espace mesurable. Soit $A \subset E$ une partie de
$E$ et $i \colon A \hookrightarrow E$ l'injection canonique de $A$ dans $E$. La
plus petite tribu sur $A$ rendant $i$ mesurable est l'ensemble des parties de
$A$ de la forme $A \cap M$ avec $M \in \mathcal{T}$. Cette tribu est appelée
tribu induite sur $A$ et sera notée $\mathcal{T}_A$.
\end{prop}
\end{mandatory}
\begin{proof}
Soit $M \in \mathcal{T}$. On remarque que $i^{-1}(M)=A \cap M$. Toute tribu sur
$A$ pour laquelle $i$ est mesurable doit évidemment contenir ces parties. Comme
par ailleurs l'ensemble des $A \cap M, M \in \mathcal{T}$ est une tribu, elle
est donc la plus petite pour laquelle $i$ soit mesurable.
\end{proof}
\begin{mandatory}
\begin{prop}
Soit $(E, \mathcal{T}, \mu)$ un espace mesuré et soit $A \subset E$ une partie
mesurable de $E$. L'application $\mu_A \colon \mathcal{T}_A \to \mathbb{R}^+$
telle que pour tout mesurable $M \in \mathcal{T}_A$, $\mu_A(M) = \mu(M)$ est une mesure
sur $(A, \mathcal{T}_A)$ appelée restriction de $\mu$ à $A$.
\end{prop}
\end{mandatory}
\begin{proof}
La seule chose réellement à prouver est que $\mu_A$ est bien définie. Tout
mesurable $M \in \mathcal{T}_A$ est de la forme $M=A \cap \tilde{M}$ avec $\tilde{M}$ mesurable de $\mathcal{T}$. Si $A$ est
mesurable, toutes les parties de $\mathcal{T}_A$ sont éléments de $\mathcal{T}$,
$\mu$ est donc bien définie sur de telles parties.
\end{proof}
On notera la nécessité d'avoir $A$ mesurable pour pouvoir définir la mesure
restreinte.

On peut appliquer le critère de mesurabilité pour montrer l'importante
proposition suivante:
\begin{mandatory}
\begin{prop}
Soient $E,F$ deux espaces topologiques munis de leurs tribus de Borel. Toute
application continue $f \colon E \to F$ est également mesurable.
\end{prop}
\end{mandatory}
\begin{proof}
La tribu de Borel d'un espace topologique est engendrée par les ouverts. Si $f
\colon E \to F$ est continue, l'image réciproque de tout ouvert de $F$ est un
ouvert de $E$. $f$ vérifie donc le critère de mesurabilité.
\end{proof}
\begin{rem}
Dans le langage des probabilités, une application mesurable
s'appelle une variable aléatoire, mais il s'agit bien du même objet
mathématique.
\end{rem}

\begin{mandatory}
\begin{prop}{(mesure image)}
Soit $(E, \mathcal{T}, \mu)$ un espace mesuré et $(F, \mathcal{F})$ un
espace mesurable. Soit $f : E \to F$ mesurable. L'application $f_* \mu
: \mathcal{F} \to \overline{\mathbb{R}^+}$ définie par~:
\[
\forall A \in \mathcal{F}, \, f_* \mu (A) = \mu(f^{-1}(A))
\]
est une mesure appelée mesure image de $\mu$ par $f$.
\end{prop}
\end{mandatory}
On notera que le caractère mesurable de $f$ est indispensable pour avoir cette
propriété de transport des mesures. Attention à bien prendre garde aux ensembles
de départ et d'arrivée entre définition d'une application mesurable (utilisant
l'image réciproque) et mesure image (qui est une image directe).
\begin{exercice}
Soit $(E,\mathcal{T},\mu)$ un espace mesuré avec $\mu(E)<+\infty$.
 Soit $f\colon E \to E$ une application mesurable. 
 On dit que $f$ préserve la mesure si $\forall A \in \mathcal{T}, \mu(f^{-1}(A))=A$, et on supposera dans la suite
  que $f$ vérifie cette propriété.
Enfin, on notera, pour $n \geq 1$ et $A \subset E$: $f^{-n}(A)=\{x \in E \, | \, f^n(x)\in A\}$  
avec $f^n=\underbrace{f \circ f \dots \circ f}_n$.
\begin{enumerate}
\item Montrer que pour tout $n \geq 1$, on a $\forall A \in  \mathcal{T}, \mu(f^{-n}(A))=A$.
\item Pour $A \in \mathcal{T}$, on pose:
\[
H_A = \{ x \in A \, | \, \forall n \geq 1, f^n(x) \notin A \}
\]
Montrer que $H_A$ est mesurable.
\item Montrer que pour tout $n \geq 1$, on a $f^{-n}(H_A) \cap H_A = \emptyset$.
\item En déduire que pour tout couple d'entiers $(m,n), m \neq n$, on a:
\[
f^{-n}(H_A) \cap f^{-m}(H_A)= \emptyset
\]
\item Evaluer:
\[
\mu \left( \bigcup_{n \geq 0} f^{-n}(H_A) \right)
\]
et en déduire que $\mu(H_A)=0$. Ce résultat est connu sous le nom de théorème de récurrence de Poincaré.
\end{enumerate}
\end{exercice}

\section{Applications mesurables à valeurs réelles}
Dans cette partie et sauf mention contraire, les applications seront à
valeurs dans $\overline{\mathbb{R}}$ que l'on munira de sa tribu de Borel. 
\begin{mandatory}
\begin{prop}
Soit $(E, \mathcal{T})$ un espace mesurable. $f : E \to
\overline{\mathbb{R}}$ est  mesurable relativement à la tribu de Borel si l'une
des conditions équivalentes suivantes est vérifiée~:
\begin{itemize}
\item $\forall t \in \mathbb{R}, \{x \in E | f(x) \leq t \} \in
\mathcal{T}$.
\item $\forall t \in \mathbb{R}, \{x \in E | f(x) < t \} \in
\mathcal{T}$.
\item $\forall t \in \mathbb{R}, \{x \in E | f(x) > t \} \in
\mathcal{T}$.
\item $\forall t \in \mathbb{R}, \{x \in E | f(x) \geq t \} \in
\mathcal{T}$.
\end{itemize}
\end{prop}
\end{mandatory}
La démonstration est laissée à titre d'exercice.
\begin{rem}
Dans la proposition précédente, on peut se limiter à des $t$ rationnels, en
raison de la densité de $\mathbb{Q}$ dans $\mathbb{R}$.
\end{rem}

 Toute application croissante $f: \mathbb{R} \to \mathbb{R}$ est mesurable. En
effet, $f^{-1}(]-\infty, x [)$ est soit vide, soit réduit à un point,
soit un intervalle de $\mathbb{R}$.
\begin{mandatory}
\begin{defn}
Une application $f : E \to \overline{\mathbb{R}}$ est dite simple si
elle ne prend qu'un nombre fini de valeurs distinctes.Une application simple
mesurable est appelée application étagée.
\end{defn}
\end{mandatory}

Une application étagée est de la forme~:
\[
f : x \to \sum_{i=1 \dots n} \lambda_i 1_{A_i}(x)
\]
avec $\lambda_i \in \overline{\mathbb{R}}$ et $A_i$ mesurable (on
rappelle que la notation $1_A$ désigne l'application indicatrice de la
partie $A$, i.e. $1_A(x) = 1$ si $x \in A$, $1_A(x) = 0$ si $x \notin
A$).



\begin{defn}
Soient $f,g$ deux applications de $E$ dans $\overline{\mathbb{R}}$. On
note $f \wedge g$ l'application $x \to \min(f(x),
g(x))$ et  $f \vee g$ l'application $x \to \max(f(x), g(x))$.
\end{defn}
\begin{prop}
Si $f,g$ sont mesurables, $f\vee g $ et $f \wedge g$ sont mesurables. 
\end{prop} 
\begin{proof}
Soit $t \in \mathbb{R}$. $\{ x | (f \vee g) (x) \leq t \} = \{ x |
f(x) \leq \} \cap \{ x | g(x) \leq t \}$. Les deux ensembles étant
mesurables par hypothèse de mesurabilité de $f$ et $g$, on en déduit
$\{ x | (f\vee g) (x) \leq t \}$ mesurable. La démonstration pour $f
\wedge g$ est similaire.
\end{proof}
\begin{mandatory}
\begin{prop}\label{prop:3}
Soient $f,g$ mesurables. L'ensemble $\{ x | f(x) < g(x) \}$ est mesurable.
\end{prop}
\end{mandatory}
\begin{proof}
En vertu de la densité de $\mathbb{Q}$ dans $\mathbb{R}$, on a $f(x) <
g(x) \Rightarrow \exists q \in \mathbb{Q} | f(x) < q < g(x)$,
l'implication réciproque étant évidente. On en déduit~:
\[
\{ x | f(x) < g(x) \} = \cup_{q \in \mathbb{Q}} \{ x | f(x) < q  \}
\cap \{x | g(x) > q \}
\]
d'où la mesurabilité de $\{ x | f(x) < g(x) \}$, union dénombrable
d'ensembles mesurables.
\end{proof}
On vérifie aisément que $\{ x | f(x) \leq  g(x) \}$ et $\{ x | f(x)
=  g(x) \}$ sont des ensembles mesurables.
\begin{mandatory}
\begin{prop}
Soit $(f_n)_{n  \in \mathbb{N}}$ une famille d'applications mesurables. Les
applications suivantes sont mesurables~:
\begin{itemize}
\item $\sup_n f_n, \inf_n f_n$.
\item $\limsup_n f_n, \liminf_n f_n$.
\item $\lim_n f_n$ sur son domaine de définition.
\end{itemize}
\end{prop}
\end{mandatory}
\begin{proof}
Soit $t \in \mathbb{R}$.
\[
\{ x | (\sup_n f_n)(x) \leq t \} = \cap_n \{ x | f_n(x) \leq t \}
\]
est mesurable comme intersection dénombrable d'ensembles mesurables
(de même pour $\inf_n f_n$). De plus~:
\begin{align*}
& \limsup_n f_n = \inf_k \sup_{n
\geq k} f_k \\
&\liminf_n f_n = \sup_k \inf_{n \geq k} f_k
\end{align*}
 et le résultat
précédent montre que ces applications sont mesurables. Le domaine de
définition de $\lim_n f_n$ est l'ensemble $D$ des points pour lesquels
$\limsup_n  = \liminf_n$ et est donc mesurable en vertu de la
proposition \ref{prop:3}. Comme par ailleurs~:
\[
\{ x | \lim_n f_n \leq t \}  =  D \cap \{ \limsup_n f_n \leq t \}
\]
on a la mesurabilité de $\lim_n f_n$.
\end{proof}
Cette proposition montre que la mesurabilité est préservée par passage
à la limite {\em simple}. La puissance de la théorie de l'intégration
au sens de Lebesgue réside dans cette propriété.
\begin{mandatory}
\begin{prop}
Soient $f,g$ des applications mesurables. Pour tout réel $\lambda$, l'application $\lambda f + g
$ est mesurable.
\end{prop}
\end{mandatory}
\begin{proof}
Soit $t  \in \mathbb{R}$. On suppose tout d'abord $\lambda \geq 0$.
\[
\{ x | \lambda f(x) + g(x) < t \}  = \cup_{q \in \mathbb{Q}} \{ x |
f(x) < q \lambda^{-1} \} \cap \{ x | g(x) < t - q\}
\]
Cet ensemble est donc mesurable.
Si $\lambda < 0$, on modifie l'expression précédente par~:
\[
\{ x | \lambda f(x) + g(x) < t \}  = \cup_{q \in \mathbb{Q}} \{ x |
f(x) > q \lambda^{-1} \} \cap \{ x | g(x) < t - q\}
\]
qui reste mesurable.
\end{proof}
\begin{mandatory}
\begin{prop}
Le produit de deux applications mesurables est mesurable, de même que
le rapport de deux applications mesurables sur son domaine de définition.
\end{prop}
\end{mandatory}
\begin{proof}
Soit $f$ mesurable. Pour tout réel $t$~:
\[
\{x | f^2(x) < t \} = \{ x | -\sqrt(t) < f(x) < \sqrt(t) \}
\]
qui est mesurable, donc $f^2$ est mesurable. 
Soit $g$ mesurable. On peut écrire~:
\[
fg = \frac{1}{2} [(f+g)^2 -f^2 - g^2]
\]
d'où $fg$ mesurable.
L'ensemble $D = \{x | g(x) \neq 0 \} = \{ x | g(x) < 0 \} \cap \{x |
g(x) > 0\}$ est mesurable. Le quotient $f/g$ a donc un domaine de
définition mesurable. Par ailleurs, pour tout $t \in \mathbb{R}$~:
\begin{align*}
\{x \in D | \frac{f(x)}{g(x)} < t \} = & \left ( \{x \in D | f(x) < t g(x) \}
\cap \{x \in D | g(x) > 0\} \right ) \\
& \cup \left (\{x \in D | f(x) > t g(x) \}
\cap \{x \in D | g(x) < 0 \} \right )
\end{align*}
l'ensemble $\{x \in D | \frac{f(x)}{g(x)} < t \}$ est donc mesurable.
\end{proof}
\begin{exercice}
Soit $f \colon \mathbb{R} \to \mathbb{R}$ une application dérivable. On suppose
$\mathbb{R}$ muni de sa tribu de Borel. 
\begin{itemize}
  \item Montrer que $f$ est mesurable.
  \item Montrer que pour tout $\tau \in \mathbb{R}$, l'application $f_\tau
  \colon x \in \mathbb{R} \mapsto f(x+\tau)$ est mesurable.
  \item En déduire que la dérivée $f^\prime$ de $f$ est mesurable. 
\end{itemize}
\end{exercice}
\begin{mandatory}
\begin{prop}\label{ch2:1}
Soit $f : E \to \overline{\mathbb{R}^+}$ une application mesurable. Il
existe une suite croissante $(f_n)_{n \in \mathbb{N}}$ d'applications
étagées telle que $f = \lim_n f_n$.
\end{prop}
\end{mandatory}
\begin{proof}
Pour tout entier $n$, on pose:
\[
f_n = \sum_{k=1}^{n2^n} \frac{k-1}{2^n} 1_{A_{n,k}} + n 1_{B_n}
\]
avec~:
\[
A_{n,k} = \{x | \frac{k-1}{2^n} \leq f(x) < \frac{k}{2^n} \}
\]
et~:
\[
B_n = E - \cup_{k=1}^{n2^n} A_{n,k} = \{x: f(x) \geq n \}
\]
La suite $f_n$ est une suite croissante d'applications étagées de
limite $f$.
\end{proof}
Le dernier exercice proposé ici se rapproche de certaines techniques utilisées
en théorie des probabilités.
\begin{exercice}
Soit $\left(\Omega, \mathcal{T}\right)$ un espace mesurable. Soit $N \colon
\left(\Omega, \mathcal{T}\right) \to \left( \mathbb{N},
\mathcal{P}(\mathbb{N})\right)$ une application mesurable. Soit $(f_n)_{n \in
\mathbb{N}})$ une suite d'applications mesurables $f_n \colon \left(\Omega,
\mathcal{T}\right) \to \left( \mathbb{R}, \mathcal{B}(\mathbb{R})\right)$.
Montrer que les applications suivantes sont mesurables:
\begin{itemize}
  \item $\omega \in \Omega \mapsto f_{N(\omega)}(\omega)$
  \item  $\omega \in \Omega \mapsto \sum_{n=0}^{N(\omega)}f_n(\omega)$
\end{itemize}
\end{exercice}
