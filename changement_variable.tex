\section{Changement de variable}

La formule du changement de variable par les $C^1$-difféomorphismes 
dans $\mathbb{R}^n$ est d'une grande importance tant pratique que
théorique. 
Elle est essentiellement basée sur un calcul de
changement de volume pour des cubes élémentaires, puis l'utilisation
du théorème des classes monotones pour passer à des Boréliens
quelconques.
Dans la suite on supposera toujours que l'on travaille dans
$\mathbb{R}^n$ muni de sa topologie usuelle.
\begin{prop}
Soit $f \mathbb{R}^n \to \mathbb{R}^n$ une application linéaire $f : x
\to M x$ avec $M$ matrice $n \times n$ inversible. Pour tout borélien
$B$, on a~:
\[ 
\lambda(f(B)) = |det(M)| \lambda(B)
\]
\end{prop}
On remarquera que l'application $B \to \lambda(f(B))$ est la mesure
image de $\lambda$ par $f^{-1}$.
\begin{proof}
Soit $B$ un borélien. Pour tout $x \in \mathbb{R}^n$, on a~:
\[
\lambda(f(B + x)) = \lambda(f(B) + f(x)) = \lambda(f(x))
\]
On en déduit que la mesure image de $\lambda$ par $f^{-1}$ est
invariante par translation, donc égale, à une constante multiplicative
près, à la mesure de Lebesgue~:
\[
\exists C >0 \, | \, \forall B \in \mathcal{B}(\mathbb{R}^n), \, \mu(f(B)) = C \mu(B)
\]
Supposons dans un premier temps $M$ orthogonale. La boule unité est
laissée invariante par $f$, d'où $C = 1 = |det(M)|$ dans ce cas. Supposons
maintenant que $M$ est diagonale, à coefficients diagonaux $\sigma_i,
i=1\dots n$ positifs. L'image de $[0,1]^n$ est $\prod_{i=1\dots n}
[0, \sigma_i]$ on en déduit $C = \prod_{i=1\dots n} \sigma_i =
|det(M)|$. Dans le cas général, on applique les deux résultats
précédents à la décomposition en valeurs singulières $M = U \Sigma
V^t$ de $M$.
\end{proof}
\begin{defn}
Une application inversible $f : \mathbb{R}^n \to \mathbb{R}^n$ est
appelée $C^k$-difféomorphisme si elle est de classe $C^k$ ainsi que
son inverse.
\end{defn}
\begin{term}
Soit $f$ une application différentiable . L'application
jacobienne (ou Jacobien) est l'application $J_f : x \to det(f^\prime(x))$. 
\end{term}
\begin{prop}
Soit $U,V$ deux ouverts de $\mathbb{R}^n$ et $K$ un compact de $U$. Soit $f : U \to V$ un
$C^1$-difféomorphisme. Pour tout $\epsilon > 0$, il existe $\delta >
0$ tel que pour tout cube $C$ de centre $x_0 \in K$ dont la longueur du
côté est inférieure à $\delta$~:
\[
(1-\epsilon)^n |J_f(x_0)| \lambda(C) \leq \lambda(f(C)) \leq (1+\epsilon)^n
|J_f(x_0)| \lambda(C)
\]
\end{prop}
\begin{proof}
Par hypothèse, $f^\prime$ est continue, il existe donc $\delta > 0$
tel que~:
\[
\| x - x_0 \| < \delta \rightarrow \| f^\prime(x) - f^\prime(x_0) \| < \epsilon
\]
On peut choisir $\delta$ suffisamment petit pour que  $C \subset K$
et que la propriété ci-dessus soit vraie. 
Pour tout $x \in C$, il existe $c \in C$ tel que $f(x) - f(x_0) =
f^\prime(c) (x-x_0)$. En combinant ce résultat avec le précédent, on
en déduit~:
\[
\| f(x) - f(x_0) - f^\prime(x_0)(x-x_0)\| \leq \epsilon \|x-x_0\|
\]
Soit l'application affine~:
\[
T : x \to f(x_0)+f^\prime(x_0)(x-x_0)
\]
On peut écrire, pour $x \in C$~:
\[
f(x) = T(x + T^{-1}g(x,x_0))
\]
avec $\| g(x,x_0) \| \leq \epsilon \|x -x_0\|$.
En posant~:
\[
\eta = \sup \{ \| f^{\prime}(x)^{-1} \|, x \in K \}
\]
il vient~:
\[
\|  T^{-1}g(x,x_0) \| \leq \eta \epsilon \| x - x_0 \|
\]
et donc~:
\[
\lambda(f(C)) \leq \lambda(T((1+\eta \epsilon )C)) =
(1+\eta\epsilon)^n |J_f(x_0)| \lambda(C)
\]
On montrerait de même la minoration.
\end{proof}
\begin{defn}
Un cube élémentaire d'ordre $k \geq 1$ est un cube de la forme~:
\[
C = \prod_{i=1}^n [ x_i 2^{-k}, (x_i + 1)2^{-k} ]
\]
où les $x_i, i=1\dots n$ sont des entiers.
\end{defn}
En recouvrant un cube élémentaire $C$ d'ordre $k$ quelconque par des
cubes vérifiant les hypothèses de la proposition précédente pour
$\epsilon > 0$ donné, on obtient (exercice)~:
\[
(1-\epsilon)^2 \int_{C} |J_f(x)|d \lambda(x) \leq \lambda(f(C)) \leq
(1+\epsilon)^2 \int_{C} |J_f(x)| d \lambda(x)
\]
puis l'égalité par passage à la limite. 
\begin{prop} Changement de variable
Soient $U,V$ ouverts de $\mathbb{R}^n$. Soit $f : U \to V$ un
$C^1$-difféomorphisme. On a, pour tout borélien $B$~:
\[
\lambda(f(B)) = \int_B |J_f(x) |d \lambda(x)
\]
\end{prop}
\begin{proof}
Les cubes élémentaires d'adhérence incluse dans $U$ forment une
classe monotone $\mathcal{C}$. La mesure  $\lambda(f(C))$ d'un cube
$C$ de cette classe est égale à $ \int_C |J_f(x) |d \lambda(x)$. Le
résultat se déduit du théorème des classes monotones en remarquant que
$\mathcal{C}$ est stable par intersections finies et engendre la tribu
de Borel.
\end{proof}
\begin{rem}
 L'extension de la proposition au cas des $C^1$-difféo\-morphismes par
morceaux se fait immédiatement en l'appliquant sur une partition de
$U$.
\end{rem}
\begin{prop}
Sous les hypothèses précédentes, on a, pour toute application $g : V
\to \mathbb{R}^+$~:
\[
\int_V f(x) d \lambda(x) = \int_U g(f(x)) |J_f(x)| d \lambda(x)
\]
\end{prop}
\begin{rem}
La proposition est une conséquence immédiate du résultat sur la mesure
image. Pour une application $g$de signe quelconque, on utilisera souvent
la formule ci-dessus sur $|g|$ pour montrer la sommabilité, puis à
nouveau sur $g$ pour calculer l'intégrale (en toute rigueur pour
$g^+,g^-$, mais un calcul élémentaire montrer que cela revient au même).
\end{rem}
