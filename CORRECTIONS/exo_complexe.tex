\documentclass[a4paper,12pt]{amsart}
\usepackage[french]{babel}
\usepackage[font={small,it}]{caption}
\usepackage{subcaption}
\usepackage{amssymb}
\usepackage{amsmath}
\usepackage{amsfonts}
\usepackage[utf8]{inputenc}
\usepackage[T1]{fontenc}
\usepackage{graphicx}
\usepackage{marginnote}
\usepackage{mparhack}
\usepackage{framed}
%\usepackage{times}
\theoremstyle{plain}
\newtheorem{theorem}{Théorème}
\newtheorem{corollaire}{Corollaire}
\newtheorem{lemme}{Lemme}
\newtheorem{prop}{Proposition}
\theoremstyle{definition}
\newtheorem{defn}{Définition}
\newtheorem{ax}{Axiome}
\newtheorem{ex}{Exercice}
\theoremstyle{remark}
\newtheorem{notation}{Notation}
\newtheorem{rem}{Remarque}
\newtheorem{term}{Terminologie}
\newtheorem{exemple}{Exemple}
% several common operators
\DeclareMathOperator*{\sinc}{sinc}

\begingroup
\makeatletter
\g@addto@macro\framed{%
\let\marginnoteleftadjust\FrameSep
\let\marginnoterightadjust\FrameSep
}
\makeatother
\endgroup

%%%%%%%%%%%%%%%%%%%%%%%%%%%%%%%
% Raccourcis divers
%%%%%%%%%%%%%%%%%%%%%%%%%%%%%%%
\newcommand{\rs}{\widehat{\mathbb{C}}}

%%%%%%%%%%%%%%%%%%%%%%%%%%%%%%%
% Environnement exercice
%%%%%%%%%%%%%%%%%%%%%%%%%%%%%%%
\newcounter{ExoCtr}
\newenvironment{exercice}{
\begin{leftbar}
\stepcounter{ExoCtr}{\bfseries Exercice}
\arabic{ExoCtr}
\marginnote{\includegraphics[scale=0.4]{images/k-word-icon.png}}}{\end{leftbar}}

%%%%%%%%%%%%%%%%%%%%%%%%%%%%%%%
% Environnement mandatory
%%%%%%%%%%%%%%%%%%%%%%%%%%%%%%%
\newenvironment{mandatory}{%
\begin{leftbar}\marginnote{\includegraphics[scale=0.2]{images/books-icon.png}}}{\end{leftbar}}

\title{Analyse complexe - CH 4\\ Correction des exercices 5,6}
\author{S. Puechmorel}
\date{\today}

\begin{document}
\maketitle
\section*{Exercice 5}
L'application log étant multiforme, il faut introduire une coupure et choisir une détermination. Les deux cas 
envisagés dans l'exercice sont les plus courants en pratique.
\begin{enumerate}
\item Les deux points singuliers de l'application dans $\mathbb{C}-\mathbb{R}^+$ sont $\pm i$ et il s'agit de pôles simples. 
Si $R > 1, \epsilon< 0$, les deux pôles sont intérieurs au contour d'intégration (indice +1) et on a en application du théorème des
résidus:
\[
\int_{\gamma}\frac{\log^2(z)}{1+z^2}dz = i 2 \pi \left( \text{Res}_{i}f + \text{Res}_{-i}f\right)
\]
où:
\[
f \colon  z \mapsto \frac{\log^2(z)}{1+z^2}
\]
Etant donné qu'il s'agit de pôles simples, on peut utiliser la formule du cours:
\[
\text{Res}_{z_0}=\frac{\log^2(z_0)}{2z_0}
\]
qui donne:
\[
\int_{\gamma}\frac{\log^2(z)}{1+z^2}dz = i 2 \pi (-i \pi^2) = 2 \pi^3
\]
\item La quantité $|zf(z)|$ tend vers 0 dans sur les deux arcs de cercle. Le lemme de Jordan s'applique pour 
montrer que les intégrales sur ces chemins ont aussi pour limite 0. Les intégrales sur l'axe réel
positif supérieur et inférieur 
ont pour limites respectives:
\[
\int_0^{+\infty} \frac{\log^2(x)}{1+x^2}dx , \quad -\int_0^{+\infty} \frac{\left(\log(x)+i2\pi\right)^2}{1+x^2}dx
\]
d'où l'on tire:
\[
-i 2\pi I + 4 \pi^2 \int_0^{+\infty} \frac{1}{1+x^2}dx = 2 \pi^3
\]
et finalement $I=0$, avec comme résultat auxilliaire permettant de vérifier les calculs:
\[
\int_0^{+\infty} \frac{1}{1+x^2}dx = \frac{\pi}{2}
\]
\item Les modifications à apporter au calcul précédent concernent les résidus qui sont modifiés
par la nouvelle détermination du logarithme et les intégrales le long de l'axe réel négatif.
Pour les résidus, on doit bien prendre garde au fait que l'argument de $-i$ est $-\pi/2$ et non $3\pi/2$. On trouve:
\[
\text{Res}_{i}= \frac{- \pi^2}{4}\frac{1}{2i}, \, \text{Res}_{-i}= \frac{- \pi^2}{4}\frac{1}{-2i}
\]
d'où une somme de résidus nulle.
Le lemme de Jordan s'applique comme précédemment, et les deux intégrales sur l'axe réel négatif 
donnent respectivement, après changement de variable $x \to -x$:
\[
\int_0^{+\infty} \frac{\left(\log(x)+i\pi\right)^2}{1+x^2}dx , \quad -\int_0^{+\infty} \frac{\left(\log(x)-i\pi\right)^2}{1+x^2}dx
\]
On en déduit à nouveau $I=0$. En revanche, la valeur de l'intégrale auxilliaire ne peut être obtenue,
les termes en question s'annulant.
\end{enumerate}
\section*{Exercice 6}
L'exercice en question est délicat dans l'établissement de la coupure: ce point est détaillé pas 
à pas dans l'énoncé.
\begin{enumerate}
\item $g(z)=r^{\frac{1}{2}}\rho^{\frac{1}{2}}\exp(i\theta/2)\exp(i\eta/2)$ avec les angles $\theta,\eta$
appartenant à leurs domaines respectifs.
\item Les seuls points posant problème sont sur l'axe réel. Le raisonnement graphique proposé montre
que si l'on se rapproche d'un point $x$ du segment $]1, +\infty[$ par le bord supérieur, l'angle
$\theta$ tend vers 0 et l'angle $\eta$ vers $-\pi$. La limite de $g$ est dans ce cas 
$-i\sqrt{x(1-x)}$. Si 
l'on approche $x$ par le bord inférieur, l'angle $\theta$ tend vers  $2\pi$ et l'angle $\eta$ vers 
$\pi$. $g$ admet pour limite $\sqrt{x(1-x)}\exp(i\pi)\exp(i\pi/2)$ qui est encore
égale à $-i\sqrt{x(1-x)}$. On en déduit qu'il est possible de prolonger $g$ sur le
segment $]1,+\infty[$. 
\item Le même raisonnement s'applique au segment $]-\infty,0[$ la limite en $x$
est dans ce cas $i\sqrt{x(1-x)}$. En revanche, on notera que si l'on approche un point $x$ du segment $]0,1[$ par le bord supérieur $g$ tend vers $\sqrt{x(1-x)}$ et par le bord inférieur vers 
$-\sqrt{x(1-x)}$.
\item Immédiat avec la remarque précédente et une application du lemme de Jordan sur les arcs de cercle.
\item La seule difficulté réside dans l'évaluation des résidus. Comme les points singuliers sont
des pôles simples, le calcul se fait avec la formule du cours. On a:
\begin{align*}
& \text{Res}_{\exp\left(\frac{i \pi }{3}\right)}f=\frac{\exp\left(-i2\pi/3\right)}{3} \\
&\text{Res}_{-1}f = i\frac{\sqrt{2}}{3} \\
& \text{Res}_{\exp\left(\frac{i \pi }{3}\right)}f=\frac{\exp\left(i2\pi/3\right)}{3}
\end{align*}
avec:
\[
f \colon z \mapsto \frac{g(z)}{1+z^3}
\]
La somme des résidus donne après simplification:
\[
i \frac{1}{3} \left(\sqrt{2}-\sqrt{3}\right)
\]
La valeur de l'intégrale s'en déduit directement.
\end{enumerate}

\end{document}
